\documentclass[12pt]{book}
\pagestyle{headings} % для более удобной навигации
\usepackage[utf8]{inputenc} % чтобы не глючило. Собирается через pdfTeX!!!
\usepackage[english,russian]{babel} % 2 языка
\usepackage[margin=2cm]{geometry} % стандартная ширина поля в книгах
\usepackage{setspace} % полуторный интервал
\usepackage{tcolorbox} % будет полезно
\usepackage{color} % будет полезно
\usepackage{droid} % вполне неплохо смотрится
\usepackage{graphicx}
\usepackage{pdfpages} % для вставки pdf при необходимости
\usepackage{hyperref}
\usepackage{indentfirst} % для отступов в основном тексте

%%% Поле для настройки отдельных %%%
%%  Элементов	ДО текста         %%
%\renewcommand{\abstractname}{О Книге}
%%%%%%%%%%%%%%%%%%%%%%%%%%%%%%%%%%%%
\begin{document}
%%% Поле для настройки отдельных %%%
%%  Элементов	по ходу           %%

%%% Оформление титульного листа %%%%
%   Титульный лист будет присое-  %%
%   диняться к файлу через pdf-   %%
%   shuffler, чтобы сработали     %%
%   все необходимые ссылки        %%
%%%%%%%%%%%%%%%%%%%%%%%%%%%%%%%%%%%%

%%%   Страница с информацией     %%%
\pagestyle{empty}
\textbf{Мир Linux}

\phantom{}
\textbf{Свен Вермален}

\phantom{}
Copyright © 2009-2013 Sven Vermeulen

\phantom{}
Перевод с английского: Галым Керимбеков\footnotetext{\href{kerimbekov.galym@yandex.ru}{$^{*}$kerimbekov.galym@yandex.ru}}$^{*}$

\phantom{}
Пост-верстка: Stas Bookman\footnotetext{\href{nodor4mint@yandex.ru}{$^{**}$nodor4mint@yandex.ru}}$^{**}$
%%% Аннотация %%%

\phantom{}

\phantom{}
\begin{tcolorbox}[colback=blue!12!white]
\noindent Книга "Мир Linux"{} предлагает мягкое, но не лишенное технических аспектов (с точки зрения конечного пользователя) знакомство с операционной системой на примере дистрибутива Gentoo Linux. Книга не станет описывать историю ядра Linux или дистрибутивов, либо погружать в детали, менее интересные для пользователей.

\phantom{}
\noindent Онлайн-руководство Gentoo предлагает весьма подробный подход по ряду разделов и поэтому обязательно к чтению любым пользователем, желающим знать обо всех возможностях этой операционной системы. Несмотря на то, что издание "Мир Linux"{} и онлайн-руководство Gentoo определённо пересекаются между собой, книга ни в коем случае не призвана заменить последнее.

\phantom{}
\noindent "Мир Linux"{} попытается сосредоточиться на темах, о которых повседневные пользователи, вероятно, должны знать, чтобы продолжать работать с Gentoo Linux.

\phantom{}
\noindent Версия, которую вы читаете в настоящее время, является v1.17 и была сгенерирована 18.02.2016 г. Доступны также версии ODT и ePUB.
\end{tcolorbox}

%%%  Оглавление  %%%
\tableofcontents %%%
%%%%%%%%%%%%%%%%%%%%

\newpage

%%%  Введение  %%%

\onehalfspacing % дальше по тексту -- полуторный интервал
%\pagestyle{headings} % С номерами страниц и заголовками

\section*{Введение}

В области настольных графических сред, Linux, предположительно, небольшой игрок (рыночные исследования оценивают долю на рынке приблизительно в 3\%). Однако наиболее вероятно, что вы знаете двух или более человек, использующих Linux, некоторые из них даже исключительно. Если принять это во внимание, то либо вы знакомы с предпочтениями ОС у более, чем ста человек, либо данную статистику стоит рассматривать с некоторым сомнением.

Тем не менее, 3\% - это все ещё много (Вы когда-нибудь думали о том, насколько много настольных систем? Я никогда не находил ответа на  этот вопрос). И если мы примем во внимание другие рынки (встраиваемые системы, серверы, сетевые устройства и прочее), доля Linux увеличится.

Но тем не менее, множество людей понятия не имеют, что такое Linux или как с ним работать. В этой книге я предлагаю техническое, краткое введение в операционную систему  с пользовательской точки зрения. Я не собираюсь погружаться в понятия, преимущества или недостатки Свободного Программного обеспечения (тем не менее, несколько абзацев не станут болезненными) и рассказывать об истории и развитии операционных систем Linux. Для ознакомления с большим числом ресурсов по этим темам, обратитесь к разделу «Дальнейшие ресурсы» в конце этой главы.

Чтобы стать полноценной книгой о Linux как об операционной системе, важно сообщить пользователю об операционных системах в целом. Linux очень модульный и открытый, и это подразумевает, что пользователю виден каждый компонент в системе. Без понимания структуры ОС пользователю было бы трудно осмыслить предназначение каждого модуля. Поэтому, я посвящаю весь раздел операционным системам.

Как только мы дойдём до задач операционной системы, я продолжу рассказ о реальных системах Linux: дистрибутивах.

В завершение, каждая глава в этой книге предложит ряд упражнений, которые можно попытаться решить. Вы не сможете найти в этой книге ответы по каждому вопросу. Предпочтительнее было бы взглянуть на упражнения, как на средство для дальнейшего продвижения и помощь в поиске (и нахождении) больше тем, связанных с Linux. В конце книги приведён список подсказок и/или ответов по вопросам.

\newpage

%%% Глава 1: Анатомия операционной системы %%%
\pagestyle{headings}
\chapter{Анатомия операционной системы}

Операционная система --- фактически стек программного обеспечения, каждый элемент, разработанный для определённой цели.

\begin{itemize}
	\item \textbf{Ядро системы}: управляет обменом данных между устройствами и программным обеспечением, системными ресурсами (такие как процессорное время, память, сеть...) и экранирует разработчика от сложности программирования устройства, предоставляя программисту интерфейс для управления аппаратными средствами.
	\item Системные библиотеки содержат программные методы для разработчиков, необходимые для написания программного обеспечения для операционной системы. Они включают в себя методы создания и манипулирования, обработки файлов, сетевого программирования, и т.д. Это - жизненно важная часть операционной системы, так как вы не можете (или не  должны) связываться с ядром напрямую: библиотека экранирует системного программиста от сложности программирования ядра.
	\item \textbf{Системные инструменты} собраны с использованием системных библиотек и позволяют администраторам следить за системой: управлять процессами, перемещаться по файловой системе, запускать другие приложения, настраивать сеть\ldots
	\item \textbf{Инструменты разработки} предоставляют средства для сборки нового программного обеспечения в (или для) системе. Несмотря на то, что это необязательная часть операционной системы, мне очень нравится упоминать их, так как с Gentoo их наличие является требованием (почему дело обстоит так, мы узнаем позже). Эти инструменты включают в себя компиляторы (переводят код в машинный), компоновщики (собирают машинный код и объединяют его в рабочий двоичный файл) и средства, значительно упрощающие процесс сборки.
\end{itemize}

Другие находящиеся в системе библиотеки улучшают опыт написания кода разработчиками, обеспечивая доступ к методам, которые уже написаны другими. Примеры таких библиотек включают в себя графические (для управления окнами) или научные библиотеки. Их наличие не обязательно в каждой системе, но если вы захотите запустить определённый инструмент, это потребует установку соответствующих библиотек. Поверх этих дополнительных библиотек вы обнаружите готовые к установке инструменты конечного пользователя (пакеты офисных программ, мультимедийные утилиты, графические среды\ldots).

\section{Ядро}

Как правило, ядро имеет четыре основные обязанности:
\begin{itemize}
	\item управление устройствами
	\item управление памятью
	\item управление процессами
	\item обработка системных вызовов
\end{itemize}

Первая из них называется управление устройствами. Компьютерная система имеет несколько соединённых с ней устройств: доступны не только ЦП и память, но также и диски (и дисковые контроллеры), сетевые платы, графические карты, звуковые карты... Так как каждое устройство работает по-разному, ядру необходимо знать, что может делать устройство, как адресовать и управлять каждым из устройств, обеспечивая корректное функционирование в  системе. Эта информация хранится в драйвере устройства: без такого драйвера ядро не знает об устройстве и поэтому не сможет управлять им.

Наряду с драйверами устройств, ядро также управляет обменом данных между ними: оно управляет доступом к совместно используемым компонентам так, чтобы все драйверы могли работать как единое целое. Весь обмен должен придерживаться строгих правил, а ядро подтверждать их соблюдение.

Компонент управления памятью управляет использованием памяти системой: оно отслеживает используемую и неиспользованную память, присваивает процессам,  требующим её, и гарантирует то, что они не смогут управлять данными друг друга. Чтобы осуществить это, ядро использует понятие адресов виртуальной памяти: адреса для одного процесса не являются действительными, и ядро отслеживает корректную картографию этих адресов. Также возможно, что в действительности данные отсутствуют в памяти, несмотря на то, что они присутствует для процесса: такие данные хранятся в области подкачки. Поскольку область подкачки намного медленнее реальной памяти, использование этого пространства должно быть ограничено данными, которые считываются нерегулярно.

Чтобы гарантировать, что каждый процесс получает достаточно процессорного времени, ядро отдаёт приоритеты процессам и даёт каждому из них определённое количество процессорного времени, прежде чем остановит процесс и передаст ЦП следующему. Управление процессами имеет дело не только с делегацией процессорного времени (называется планированием), но также и с полномочиями безопасности, информацией о владельце процесса, обменом данными между процессами и т.д.

Наконец, чтобы фактически работать над системой, ядро должно обеспечиваться средствами, необходимыми для системы и программиста, чтобы контролировать само себя, предоставлять или получать информацию, исходя из которой могут быть приняты новые решения. Используя системные вызовы программист может написать приложения, которые запрашивают ядро для получения информации или просят, чтобы оно выполнило определённую задачу (например, управлять некоторыми аппаратными средствами и возвращать результат). Конечно, такие вызовы должны быть безопасны в использовании,  чтобы вредоносные программисты не могли привести систему в нерабочее состояние  хорошо обработанным вызовом.

Операционная система, такая как Gentoo Linux, использует Linux в качестве ядра.

\section{Системные библиотеки}

Поскольку ядро само по себе мало на что способно, для выполнения задач оно должно инициироваться триггерами. Такие триггеры создаются приложениями, однако эти приложения  конечно должны знать, как устанавливать системные вызовы ядра. Поскольку каждое ядро имеет различный набор системных вызовов (очень специфичных по отношению к системе), программисты создали стандарты с которыми они могут работать. Каждая операционная система поддерживает эти стандарты, которые в дальнейшем переводятся в соответствующие системные вызовы для этой системы.

Стандарт в качестве примера - библиотека Cи, вероятно наиболее важная доступная системная библиотека. Эта библиотека делает доступными довольно первостепенные операции для программиста, такие как основная поддержка ввода/вывода, строковые подпрограммы обработки, математические методы, управление памятью и операции с файлами. С помощью этих функций программист может создать программное обеспечение, которое основывается на каждой операционной системе, поддерживающей библиотеку Cи. Эти методы далее  переводятся библиотекой Cи в определённые системные вызовы ядра (если они необходимы). Таким образом, у программиста нет необходимости знать о внутренностях ядра, он даже может написать программу (однократно), которая может быть собрана для множества платформ.

Не существует какой-либо единственной спецификации по тому, какова системная библиотека. Автор этой книги полагает, что системные библиотеки — это любые библиотеки, являющиеся частью стандартной, минимальной инсталляции операционной системы. Также системные библиотеки для одной операционной системы (и даже дистрибутив) могут отличаться от аналогичных библиотек для другой. У большинства дистрибутивов Linux имеются те же системные библиотеки, что неудивительно, так как все они могут выполнять одно то же программное обеспечение, если конечно оно собрано для этих библиотек. Некоторые дистрибутивы просто не отмечают одну библиотеку как часть стандартной, минимальной инсталляции, в отличие от других.

Самая распространённая системная библиотека для систем Linux - GNU Cи, также известная как glibc.

\section{Системные инструменты}

Так же как и с системными библиотеками, не существует единой спецификации для системных инструментов. Тем не менее, в отличие от системных библиотек они достаточно прозрачны для конечного пользователя. Поэтому почти все дистрибутивы Linux используют те же системные инструменты или подобные инструменты с  аналогичными функциями, но реализованные иначе.

Но что такое системные инструменты? Ну, вы все ещё не можете управлять своей системой с помощью ядра и некоторых библиотек программирования. Для этого вам необходим доступ к командам для ввода в интерпретированную и запущенную систему. Эти команды выполняют простейшие задачи, такие как навигация по файлам (переход в каталог, создание/удаление файлов, получение файла листингов...), управление информацией (текстовый поиск, сжатие, перечисление различий между файлами...), управление процессом (запуск  новых процессов, получение листинга процессов, завершение рабочих процессов...), задачи, связанные с полномочиями (изменение владельца файлов, изменение идентификаторов пользователей, обновление разрешений файлов...) и др.

Если вы не знаете, как иметь дело со всем этим материалом, то не имеете навыков работы с системой. Некоторые операционные системы скрывают эти задачи за усложнёнными средствами, другие имеют простые, но отдельные для каждой задачи и объединяют мощь всех этих инструментов. Unix (и Linux) является одним из последних. Системы Linux для большинства этих задач обычно имеют набор системных утилит GNU (GNU Core Utilities).

\section{Средства разработки}

С упомянутыми тремя компонентами у вас есть запущенная, рабочая  система. Возможно вы не сможете сделать всё, что захотите, однако можно обновить систему до того состояния, прежде чем она станет выполнять желаемое. Каким образом? Устанавливая дополнительные инструменты и библиотеки до тех пор, пока у вас не будет своей функционирующей системы.

Эти дополнительные инструменты и библиотеки, несомненно, написаны программистами, а это значит, что они должны быть способны собрать свой код так, чтобы он работал в системе. Некоторые системы, такие как Gentoo Linux, даже собирают их вместо того, чтобы полагаться на исходный код, предварительно собранный другими людьми. Чтобы собрать эти инструменты, необходим исходный код каждого из них,  а также инструментов, необходимых для преобразования исходного кода в исполняемые файлы.

Называются они «Инструментарий разработки» (tool chain): набор связанных между собой инструментов, необходимых для разработки рабочего приложения. Общий набор инструментов разработки состоит из текстового редактора (для написания кода), компилятора (для преобразования кода в специфичный для машины язык), компоновщика (для объединения машинного кода нескольких исходных текстов - включая предварительно собранные "совместно используемые" (shared) библиотеки в единый исполняемый файл) и библиотек (их я просто упоминал как "совместно используемые").

Этот инструментарий имеет наибольшее значение для разработчика; это — жизненно важное средство разработки, но не единственное. К примеру, разработчики графических приложений обычно нуждаются в инструментах для создания графики, или даже инструментах, связанных с мультимедией, для добавления звуковых эффектов в их программу. Средство разработки - общее существительное для инструмента, необходимого разработчику для создания чего-либо, но несущественного для операционной системы пользователя среднего уровня, который не является разработчиком.

Наиболее известные средства разработки также предоставляются Фондом GNU, а именно, набор компиляторов GNU (GNU Compiler Collection), также известный как gcc.

\section{Инструменты, предоставляемые конечному пользователю}

Как только разработчик закончит создание своего продукта, у вас появится инструмент конечного пользователя с сопроводительными библиотеками (которые могли бы потребоваться другим инструментам, который представляют собой модифицированные сборки на основе этого продукта). Инструменты, делающие систему уникальной для пользователя, представляют собой то, что он хочет сделать с ней. Несмотря на невостребованность самой  операционной системой, они необходимы конечному пользователю и поэтому очень важны для его системы.

Большинство операционных систем не устанавливает все инструменты или большинство их них по причине того, что их слишком много чтобы выбрать. Некоторые операционные системы даже не обеспечивают пользователя средствами их установки, но полагаются на изобретательность программиста, на создание им установщика, который интегрирует инструмент в систему. Иные  системы поставляют с собой небольшое, но значительное подмножество инструментов конечного пользователя, чтобы их пользователи могли быстро обновить свои системы в любом виде, каком они пожелают, не требуя долгого и изнурительного поиска через Интернет (или ещё хуже, магазин аппаратного или программного обеспечения)  программ, в которых они нуждаются.

Примеры инструментов конечного пользователя известны как пакеты офисных программ, инструменты графического дизайна, мультимедийные проигрыватели, коммуникационное программное обеспечение, интернет-браузеры.

\section{Хорошо, что такое этот GNU?}

Проект GNU - усилие нескольких программистов и разработчиков по созданию свободной,  Unix-подобной операционной системы. GNU является рекурсивным акронимом, означающим, что GNU не Unix, поскольку она подобна ему, но не содержит его кода и (и остаётся) является свободной. Фонд GNU, юридическое лицо, стоящее за проектом GNU, видит свободу как нечто большее, чем просто свободу в финансовом значении: программное обеспечение должно быть свободным для использования в любых целях вообще,  для изучения,  изменения исходного кода ПО и его поведения, копирования и свободного распространения изменений, внесённых вами.

Данная идея свободного программного обеспечения - благородная мысль, активная в умах многих программистов: следовательно, множество тайтлов программного обеспечения находится в свободном доступе. Программа обычно сопровождается лицензией, которая объясняет, что вы можете и не можете делать с ней (также известное как "Лицензионное соглашение для конечного пользователя" - EULA). Свободное программное обеспечение также имеет подобную лицензию - в отличие от EULA, вместо того, чтобы запрещать она фактически многое позволяет. Пример такой лицензии - GPL - Универсальная общедоступная лицензия GNU.

\section{Linux в качестве ядра операционной системы}

Когда мы смотрим на операционную систему Linux, её основным компонентом является ядро. То самое, которое используют все подобные системы - ядро Linux или просто Linux. Да, правильно, систему Linux называют по имени ядра.

Теперь несмотря на то, что все операционные системы Linux используют одноимённое ядро, многие из них используют различные адаптированные версии. Это вызвано тем, что  его разработка имеет несколько ответвлений. Самым важным я назову ванильное ядро. Основное ядро разработки, где  продолжает работать большинство его разработчиков; любое другое основывается на нём. Другие ядра предоставляют функции, которые ванильное ядро  ещё не имеет (или отказалось в пользу других); тем не менее, они полностью с ним совместимы.

Ядро Linux впервые увидело  свет в 1991 году, разрабатывается (и все ещё актуально) Линусом Торвальдсом. Развивалось оно стремительно (версия 1.0.0 уже в 1994 г.) как в размере (версия 1.0.0 имела более чем 175000 строк кода), так и в популярности. За эти годы его модель разработки осталась прежней: существуют немного крупных игроков в разработке, которые решают, что входит и что остаётся вне кода ядра, но большинство вкладов происходит от десятки сотен волонтёров (усвой вклад в ядро версии 2.6.21 несло более чем  800 человек).

Последняя версия ядра на момент написания этой книги - 3.10.7. Первые два числа играют роль основной версии, третье число - вспомогательная версия (главным образом, обозначающее выпуски с исправлениями ошибок). Иногда добавляется четвёртое число, если необходимо одноразовое исправление ошибки. При разработке ядра Linux обычно постепенно увеличиваются основные числа (большую часть времени второе число), обозначающие выпуски с функциональными улучшениями: с каждым инкрементом пользователи (и разработчики) узнают, что таким образом, ядро получает новые функции.

Как только выпускается новая версия ядра Linux, оно не распространяется всем его пользователям. Нет, это уже та область, где играют роль дистрибутивы\ldots

\section{Операционные системы Linux: дистрибутивы}

Если бы пользователь захотел установить систему Linux без дополнительной справки, ему пришлось бы самостоятельно собрать ядро, компоненты операционной системы (библиотеки, конечные инструменты...) и отслеживать изменения в мире свободного программного обеспечения (такие как выход новых версий или исправлений безопасности). И несмотря на то, что всё это совершенно возможно (обратитесь к проекту Linux From Scratch), большинство пользователей пожелало бы что-нибудь более... удобное для них.

Откройте для себя дистрибутивы. Проект дистрибуции (такой как Проект Gentoo) ответственен за расширение системы Linux (дистрибутива) для конечного пользователя, по сути точка контакта для её установки.

Проекты дистрибуции делают выбор относительно программного обеспечения:

\begin{itemize}
	\item[-] Каким образом пользователи должны установить операционную систему?
\end{itemize}

Возможно, пользователи поощряются к выполнению как можно больше шагов в  процессе установки ("дистрибутив" Linux From Scratch, вероятно имеет самый интенсивный процесс установки). Наиболее противоположным этому является загрузочный CD или USB образ системы, которая даже не требует какой-либо настройки или установки: она просто загружает среду и вы можете начать использовать такой дистрибутив.

\begin{itemize}
	\item[-] Какие имеются варианты установки (CD, DVD, сеть, Интернет\ldots?)
\end{itemize}

Большинство дистрибутивов Linux предлагает вариант установки с CD/DVD, поскольку это самый популярный способ для получения программного обеспечения. Однако существуют множество других вариантов. Можно установить дистрибутив по сети, используя начальную сетевую загрузку (популярный подход в корпоративной среде, так как он делает возможной установку без сопровождения), или из другой операционной системы.

\begin{itemize}
	\item[-] Какое программное обеспечение должно быть доступно пользователю?
\end{itemize}

Популярные настольные дистрибутивы предлагают конечным пользователям широкий диапазон программного обеспечения. Это позволяет дистрибутиву стать общепризнанным, поскольку он соответствует потребностям многих пользователей. Однако существуют более усовершенствованные дистрибутивы, которые сконцентрированы на определённый рынок (абонентские установки для мультимедийных представлений, брандмауэры и управление сетью домашние устройства автоматизации...) и конечно эти дистрибутивы предлагают пользователям различные тайтлы программ.

\begin{itemize}
	\item[-] Каким образом собрано доступное программное обеспечение (определённая система, функции\ldots)?
\end{itemize}

Если дистрибутиву требуется, чтобы программное обеспечение работало на наибольшем количестве типов процессоров (Pentium, i7, Athlon, Xeon, Itanium\ldots), необходимо  собрать программное обеспечение для универсальной платформы (скажем i686), а не для определённой (Itanium). Конечно, это означает, что программа не станет использовать все функции, которые обеспечивают новые процессоры, но в действительности оно все ещё будет работает на множестве систем.

То же самое справедливо и для функций, поддерживаемых определёнными тайтлами программ. Некоторые из них предлагают дополнительную поддержку ipv6, ssl, truetype fonts... Однако, при необходимости,  вы должны скомпилировать её в программе. Дистрибутивы, предлагающие программное обеспечение в двоичном формате (этим и занимается большинство из них) должны сделать этот выбор за своих пользователей.  Чаще всего, они пытаются предложить поддержку как можно большего количества функций, однако далеко не все конечные пользователи нуждались бы этом или даже желали бы.

\begin{itemize}
	\item[-] Так ли важна интернационализация программного обеспечения?
\end{itemize}

Некоторые дистрибутивы предназначаются для определённых групп пользователей, в зависимости от языка и географии. Существуют  полностью локализованные для определённой группы (скажем для "пользователей, говорящих на бельгийско-нидерландском языке" или "канадских франкоговорящих пользователей"), но также и пытающиеся предложить локализацию для наибольшего количества групп.

\begin{itemize}
	\item[-] Как пользователи должны обновлять и обслуживать свою систему?
\end{itemize}

Множество дистрибутивов предлагают автоматизированный, но не у всех в наличии живой процесс обновления программного обеспечения (например, когда-то установленная вами инсталляция постепенно растёт и становится последней версией дистрибутива без каких-либо определённых действий). Некоторые из них даже требуют загрузку с последнего установочного CD и выполнение шагов по обновлению.

\begin{flushright} \emph{Каким образом пользователь настраивал бы свою систему?}
\end{flushright}

Если вы --- пользователь графического варианта Linux, то определённо не хотите слышать о редактировании конфигурационных файлов или действиях, связанных с командной строки. Таким образом наиболее вероятно, вы станете искать дистрибутивы, предлагающие полный графический интерфейс для настройки системы. Тем не менее некоторым пользователям в самом деле нравится идея редактирования таких файлов напрямую, так как она предлагает наибольшую гибкость (но также и самую высокую кривую обучения) и дистрибутивы чаще всего предлагают эти предпочтения. Некоторые дистрибутивы даже не позволят обновить их напрямую, так как они  (пере)создаются без вмешательства, так или иначе перезаписывается всё, что вы изменяли.

\begin{flushright} \emph{Какова целевая группа пользователей дистрибутива?}
\end{flushright}

Большинство настольных дистрибутивов предназначены для домашних/офисных пользователей, но существуют дистрибутивы для детей или учёных. Некоторые созданы для разработчиков, а другие для людей старшего возраста, слабовидящих и не имеющих доступ к Интернету.

\begin{flushright} \emph{Какую политику дистрибутив использует в своём программном обеспечении?}
\end{flushright}

Организации, такие как FSF, имеют собственную философию касательно того, на что мир (программного обеспечения) должен быть похожим. Множество дистрибутивов предлагают способ реализовать это. Например, некоторые из них допускают лишь программное обеспечение, лицензируемое в соответствии с утверждённой FSF лицензией. Другие дистрибутивы позволяют пользователям использовать несвободное. Есть и реализующие более высокую философию безопасности, предлагая наиболее защищённый подход к операционным системам.

\begin{flushright} \emph{Дистрибутив должен быть в свободном доступе?}
\end{flushright}

Конечно, деньги --- нередко основная причина принятия решений. Не все дистрибутивы свободно загружаемы/доступны в Интернете, несмотря на то, что большинство из них таково. Но даже когда дистрибутив находится в свободном доступе, все ещё есть необходимость получить коммерческую поддержку, даже только для обновлений безопасности дистрибутива.

Вы обнаружите несколько дистрибутивов; каждый из это проектов отвечает на вопросы, которые немного отличаются от других. Следовательно, выбор правильного дистрибутива часто является задачей, в которой вы должны ответить на многие вопросы, прежде чем найти подходящий.

Конечно, когда вы только начинаете работать с Linux, у вас скорее всего ещё нет твёрдого мнения касательно этих вопросов. Это хорошо, так как если вы хотите  использовать Linux, то должны начать с дистрибутива, который обеспечивает наилучшую поддержку. Расспросите кого-нибудь рядом, возможно у вас есть друзья, которые могли бы помочь. И будем честны, что может быть лучше персональной поддержки?

\section{Что такое дистрибутив?}

Дистрибутив --- это набор программного обеспечения (названного пакетами), объединённого вместе в единый набор, который создаёт полностью функциональную среду. Пакеты содержат тайтлы программного обеспечения (сборка других проектов) и возможно содержат исправления (обновления), специфичные для дистрибутива для лучшей интеграции пакетов или гармоничности со всей средой. Эти пакеты обычно содержат не только копию выпусков, созданных другими проектами программного обеспечения, но и большую логику с целью приспособить программное обеспечение к  философии дистрибутива.

Возьмите к примеру KDE. KDE - (графическая) настольная среда, связывающая вместе несколько десятков инструментов меньшего размера. Некоторые дистрибутивы предлагают своим пользователям оригинальную инсталляцию KDE, другие немного изменяют ее для достижения индивидуального оформления по умолчанию и т.п.

Другим примером был бы MPlayer, мультимедийный проигрыватель, наиболее известный широкой поддержкой различных видеоформатов. Однако, если вы хотите просмотреть видеофайлы Windows Media (WMV), в этом случае потребуется встроенная поддержка (несвободных) win32 кодеков. Некоторые дистрибутивы предоставляют её для MPlayer, другие нет. Gentoo Linux позволяет выбирать, хотите вы её или нет.

\section{Что обеспечивает дистрибутив?}

Если вы хотите использовать дистрибутив, то можете (но не обязаны) использовать инструменты, созданные проектом дистрибутива для упрощения нескольких задач:

\begin{itemize}
	\item для установки дистрибутива, можно использовать одну или более программ установки, обеспечиваемых проектом
	\item чтобы установить дополнительные пакеты в системе, можно использовать один или несколько инструментов управления программным обеспечением, обеспечиваемых проектом
	\item для настройки системы можно использовать один или несколько инструментов настройки, обеспечиваемых проектом.
\end{itemize}

Я не могу не подчеркнуть достаточную важность термина. Вы не обязаны использовать программу установки дистрибутива (всегда можно установить дистрибутив по-другому), и не обязаны устанавливать программы, используя инструменты управления программным обеспечением (можно собирать и устанавливать вручную), и не обязаны настраивать систему инструментами  для настройки (вы можете вручную редактировать конфигурационные файлы различных приложений).

Почему в таком случае дистрибутив перекладывает все эти усилия на инструменты? Причина заключается в том, что они намного упрощают использование системы для пользователя.  Возьмите в качестве примера установку программ. Если вы не используете инструмент управления программным обеспечением, то должны собрать программу самостоятельно (что может отличаться в зависимости от программы, которую вы хотите собрать), отслеживать обновления (исправления ошибок и безопасности), удостовериться в том, что установили все зависимое программное обеспечение (программное обеспечение, которое зависит от другого, что, в свою очередь, зависит от библиотеки a, b и c...), и отслеживать установленные файлы, чтобы система не переполнилась хламом.

Другое основное дополнение, которое обеспечивают дистрибутивы — это пакеты программного обеспечения. Пакет содержит программный тайтл (например браузер Mozilla Firefox) с дополнительной информацией (такой как описание программного тайтла, информация о категории, зависимостях и библиотеках...) и логику (как установить программное обеспечение, как активировать определённые модули, которые оно обеспечивает, как создать запись меню в графических средах, как его собрать, если он ещё не собран...). Это может отразиться на сложности пакета, что является одной из причин того, почему дистрибутивы обычно не могут выпустить новый пакет в день выпуска версии программы.

Однако, большая часть информации и логики для исправлений безопасности остаётся такой же, выпуск исправлений безопасности для программного обеспечения обычно приводит к быстрому выпуску проектом дистрибутива исправлений безопасности к пакету с этим ПО.

Проект дистрибутива обеспечивает следующие элементы поддержки наряду с программным обеспечением из которого состоит дистрибутив:

\begin{itemize}
	\item документация по дистрибутиву
	\item инфраструктура, откуда Вы можете загрузить дистрибутив и его документацию
	\item ежедневные обновления пакетов для нового программного обеспечения
	\item ежедневные обновления безопасности
поддержка по дистрибутиву (которая может быть в виде форумов, почтовой, телефонной или даже коммерческой договорной),
 \item \ldots
\end{itemize}

Отныне проект дистрибутива — нечто большее, чем всё это. Разработчики могут сотрудничать, объединяя все пакеты в единый проект, чтобы создать  систему, которая расширяет ряд операционных систем "коммерческого сорта". Для достижения этой цели большинство проектов дистрибутива имеет подразделения для связей с общественностью, пользовательских отношений, отношений с разработчиками, управления версиями, документации и переводов и т.д.

\section{Что такое архитектура?}

Я ещё не упоминал об архитектурах, тем не менее они важны. Позвольте мне для начала определить понятие набора инструкций.

Набор инструкций ЦП - это набор команд, который понимает определённый процессор. Эти команды выполняют множество функций, таких как арифметические функции, операции памяти и управление потоком. Программы могут быть написаны с их использованием, но обычно программисты применяют высокоуровневый язык программирования, поскольку  программа, написанная на этом языке (названном ассемблером упомянутого ЦП), может запускаться только на этом ЦП. Таким образом, ассемблер настолько низкоуровневый, что написать инструмент с помощью него не совсем просто. Инструменты, использующие ассемблер — это компиляторы (которые переводят высокоуровневый язык программирования в ассемблер), загрузчики (которые загружают операционную систему в память), и некоторые базовые компоненты операционных систем (ядро Linux имеет некоторый ассемблерный код).

Теперь у каждого типа ЦП имеется различный набор инструкций. Intel IV Pentium имеет набор инструкций, отличающийся от набора Intel Pentium III; Sun UltraSPARC III  имеет набор инструкций, отличающийся от набора Sun UltraSPARC III. Однако они очень схожи. Это вызвано тем, что они находятся в том же семействе. Центральные процессоры того же семейства понимают определённый набор инструкций. Программные инструменты, созданные для одной системы команд, работают на всех центральных процессорах того же семейства, но не могут использовать в своих интересах весь набор инструкций ЦП, на котором они работают.

Семейства центральных процессоров сгруппированы по архитектурам. Архитектура - глобальная переменная и представляет понятие всей системы; она описывает, как получает доступ к дискам, как обрабатывается память, как определяется процесс начальной загрузки. Она определяет большие, концептуальные различия между системой. Например, диапазон систем, совместимый с Intel, сгруппирован в архитектуре x86; если вы загружаете такую систему, её процесс начальной загрузки запускается с BIOS (Базовая система ввода-вывода). Системы, совместимые со Sparc Sun сгруппированы в архитектуре sparc; если Вы загружаете такую систему, процесс начальной загрузки запускается с PROM.

Архитектура важна, так как дистрибутивы Linux зачастую поддерживают несколько архитектур, и вы определённо должны знать, какую архитектуру  ваша система использует. Это по всей вероятности x86 или amd64 (оба довольно эквивалентны), однако необходимо понять, что существует также и другая архитектура. Вы найдёте инструменты, которые не поддерживаются для вашей архитектуры даже при том, что они доступны для вашего дистрибутива, или некоторые пакеты могут иметь последнюю версию в наличии на одной архитектуре, но не на других

\section{Мифы, окружающие Linux}

Linux часто расхваливается в СМИ - иногда этому есть объяснение, однако большую часть времени его нет. И хотя я ранее обсуждал, чем является Linux, напомню вкратце:

\begin{quote}
\emph{Linux - общее обозначение, относящееся к операционной системе Linux, набору инструментов, работающих под управлением ядра Linux и большую часть времени предлагаемых через проект дистрибутива.}
\end{quote}

Конечно, зачастую это не вносит ясность для пользователей, незнакомых с миром вне Microsoft Windows. Несмотря на то, что лучший способ узнать, что такое Linux, это использовать его, мне кажется важным разоблачить некоторые мифы, прежде чем продолжить остальную часть книги.

\section*{Мифы}

Миф — история, которая популярна, но не верна.  Мифы, окружающие Linux будут существовать всегда. Следующие несколько разделов попытаются предложить мои идеи касательно развенчанию большинства из них...

\subsection{Linux трудно установить}

Кто-то всегда может указать на дистрибутив, который трудно установить. "Дистрибутив" Linux From Scratch - фактически документ, объясняющий весь процесс установки дистрибутива Linux путём сборки компиляторов, программного обеспечения, размещением файлов, и т.д. Да, это тяжело и даже могло быть сложным, если бы документация не была актуальной.

Тем не менее, множество дистрибутивов (даже большинство из них) просты в установке. Они предлагают тот же подход к установке, как и другие операционные системы (включая Microsoft Windows), вместе с онлайн-справкой (экранная справка) и оффлайн-справкой (руководство по установке). Некоторые дистрибутивы могут даже быть установлены всего двумя или тремя вопросами, вы можете даже использовать Linux без необходимости  устанавливать его вообще.

\subsection{Для Linux нет поддержки }

Были дни, когда для Linux не существовало коммерческой поддержки, но это было в прошлом веке. Теперь Вы можете получить операционную систему Linux от крупных поставщиков программного обеспечения, таких как Novell или RedHat (с поддержкой), или использовать свободно загружаемый дистрибутив Linux и получить контракт с компанией, которая предлагает поддержку этого дистрибутива.

Все дистрибутивы также предлагают превосходную бесплатную поддержку (это то, о чем я расскажу в следующих нескольких главах) и у многих из них в наличии активная последующая обработка и анализ безопасности, приводящие к быстрым исправлениям, как только уязвимость находят или сообщают о ней. Часто нет никакой потребности в получении коммерческой поддержки для пользователя настольных систем, поскольку каналы поддержки в свободном доступе предлагают главное преимущество по сравнению с некоторыми другими собственническими операционными системами.

\subsection{Linux - свободное программное обеспечение, таким образом, дыры в системе безопасности находятся легко}

Так как это свободное программное обеспечение, «дырам» в системе безопасности намного труднее остаться в исходном коде. Существует слишком много глаз, наблюдающих за исходным кодом, а у многих проектов свободного программного обеспечения имеется очень активное сообщество разработчиков, которое проверяет и перепроверяет изменения исходного кода множество раз, прежде чем они будут предложены конечному пользователю.

\subsection{Linux не имеет графику}

Ядро Linux не является графическим ядром, однако инструменты, функционирующие ниже ядра, могут быть графическими. Даже больше, большинство дистрибутивов предлагает полный графический интерфейс для каждого возможного аспекта операционной системы: она загружается с графикой, вы работаете графическим методом, устанавливаете программное обеспечение графическим образом и даже диагностируете проблемы, используя графический подход. Несмотря на то, что вы можете работать исключительно с командной строкой, большинство дистрибутивов сконцентрировано на графической среде.

Эта книга не является хорошим примером относительно этого мифа, поскольку она фокусируется на командной строке. Однако это лишь из-за личных предпочтений автора.

\subsection{Я не могу запустить свою программу в Linux}

Для многих тайтлов Microsoft Windows это действительно так. Но почти наверняка существует программное обеспечение, доступное в Linux и предлагающее те же функции, что и то, к которому вы обращаетесь. Некоторые программы даже доступны для Linux: популярные браузеры Firefox и Chrome - два примера, пакет офисных программ в свободном доступе, OpenOffice.org — ещё один пример.

Существуют также эмуляторы и библиотеки, которые предлагают интерфейс, позволяющий приложениям Microsoft Windows работать в Linux. Тем не менее, я не рекомендую использовать данное программное обеспечение для каждого возможного тайтла. Это самое последнее средство в том случае, когда вам определённо требуется конкретный тайтл, но при этом вы уже выполняете большинство работы в Linux.

\subsection{Linux безопасен}

Это также миф. Linux не более безопасен, чем Microsoft Windows или Apple Mac OS X. Безопасность — это больше, чем сумма всех уязвимостей в программном обеспечении. Она основана на компетентности пользователя, администратора, конфигурации системы и др.

Linux может быть очень безопасным: существуют дистрибутивы,  фокусирующиеся на интенсивной безопасности посредством дополнительных параметров настройки, конфигураций ядра, выбора программного обеспечения и прочего. Но вы не нуждаетесь в таком дистрибутиве, если хотите иметь безопасную систему Linux. Предпочтительнее всего прочесть документацию по безопасности дистрибутива, и удостовериться в том, что вы регулярно обновляете свою систему, не запускаете программное обеспечение, в котором не нуждаетесь или не посещаете сайты, законность которых вам неизвестна.

\subsection{Linux слишком фрагментирован, чтобы когда-либо стать более крупным игроком}

Множество групп именуют Linux как фрагментируемый по причине множества дистрибутивов. Однако пользователь одного дистрибутива может легко работать с пользователями других дистрибутивов (здесь нет никакой проблемы). Пользователь одного дистрибутива может также помочь пользователям других дистрибутивов, так как их программное обеспечение - все ещё то же самое (здесь также нет никакой проблемы). Даже больше, программное обеспечение, создаваемое на одном дистрибутиве, прекрасно работает на другом дистрибутиве (здесь тоже нет никакой проблемы). Широко распространённая доступность дистрибутивов - это сила, а не слабость, поскольку она предлагает больше выбора (а также больше знаний и опыта) конечному пользователю.

Возможно, люди ссылаются на различные существующие деревья ядра Linux. Тем не менее, все эти деревья основаны на том же ведущем ядре (часто называемом как "ванильное") и каждый раз, когда ведущее ядро производит новую версию, эти деревья обновляют свой собственный код, поэтому ответвления никогда не отстают. Дополнительные деревья, которые существуют из-за целей разработки (дополнительные патчи для неподдерживаемых аппаратных средств, прежде чем они будут объединены с ведущим ядром, дополнительными патчами для определённых решений для виртуализации, которые в противном случае становятся несовместимыми или не могут быть объединены из-за проблем лицензии, дополнительные патчи, которые слишком навязчивы и требуют времени, прежде чем они будут стабилизированы, и т.д.).

Возможно, люди ссылаются на различные графические среды (такие как KDE и GNOME). Тем не менее, они умалчивают о функциональной совместимости между этими графическими средами (вы можете запустить приложения KDE в GNOME и наоборот), о стандартах, которые это разнообразие создают (стандарты работающие с форматами файлов, записями меню, связывании объектов и прочее), и прочее.

Управляемая фрагментация - это то, что предлагает Linux (и свободное программное обеспечение в целом). Управляемая, так как соответствует открытым стандартам и свободным спецификациям, которые хорошо документированы и которых придерживается все программы. Фрагментированная, так как сообщество хочет предложить  конечным пользователям больше выбора.

\subsection{Linux - альтернатива Microsoft Windows}

Linux - не альтернатива, это другая операционная система. Существует различие между значениями. Альтернативы пытаются предложить ту же функциональность и интерфейс, используя различные средства. Linux - иная операционная система, так как не стремится предложить ту же функциональность или интерфейс Microsoft Windows.

\subsection{Linux — это противовес Microsoft}

Это не так, поскольку люди, у которых есть определённые чувства к Microsoft, часто используют и Linux. Linux пытается быть, ни чем иным, как системой, полностью взаимодействующей с любой другой. Проекты программного обеспечения совершенно определённо желают, чтобы их программное обеспечение работало на любой операционной системе, не только на Microsoft Windows или Linux.

\subsection{Слабые места}

И все же не вся информация распространяется вокруг мифов. Некоторые из них - реальные слабые места, над которыми в Linux все ещё нужно продолжать работать.

\subsubsection{Поддержка игр прогрессирует медленно}

Это правда. Несмотря на то, что вокруг существует множество игр среди свободного программного обеспечения, большинство их них разработано исключительно для Microsoft Windows, и не все игры могут быть запущены в Linux с использованием эмуляторов или библиотек, таких как WINE (но к счастью, множество). Не так уж просто попросить разработчиков игр разработать их для Linux, поскольку большинство из них концентрирует свои усилия на библиотеках (такие как DirectX), доступных только для Microsoft Windows.

Тем не менее, в этой области недавно произошли улучшения. Valve выпустила Steam для Linux, обеспечивая игровой процесс в настольном Linux. Это дало большой сдвиг "играм на Linux".

Однако также появляется и другая тенденция: все больше игр  выпускается только для консолей, отбрасывая среду ПК. Я лично не знаю, как игры разовьются в будущем, но думаю, что реальные экшн-игры будут фокусироваться больше на игровые приставки.

\subsubsection{Новейшие аппаратные средства принимаются в Linux не сразу}

Если поставщик аппаратных средств не предлагает драйверы для Linux, то действительно потребуется некоторое время, прежде чем поддержка аппаратных средств будет обеспечена в ядре Linux. Однако это процесс, охватывающий не многие годы, а скорее месяцы. Возможность состоит в том, что совершенно новая видео- или звуковая карта будет поддерживаться уже в течение 3 - 6 месяцев после выпуска.

То же является истиной и для карт беспроводной сети. Принимая во внимание то, что ранее это было слабостью, поддержка карт беспроводной сети отныне хорошо скоординирована в сообществе. Основная причина этого состоит в том, что большинство поставщиков теперь официально поддерживает свой беспроводной чипсет для Linux, предлагая драйверы и документацию.

\newpage

\begin{tcolorbox}[title=\textbf{Упражнения}, colback=yellow!14!white, colframe=red!75!white]
\begin{enumerate}
	\item Создайте список дистрибутивов Linux, о которых Вы услышали, и проверьте каждый из них, как они выполняют свои задачи в сферах, которые Вы находите важными (например, доступность документации, переводов, поддержки определённых аппаратных средств, мультимедиа\ldots).
	\item Перечислите 7 архитектур ЦП.
	\item Почему новые выпуски ядра не распространяются конечному пользователю незамедлительно? Какую роль дистрибутивы играют в этом процессе?
\end{enumerate}
\end{tcolorbox}

\begin{tcolorbox}[title=\textbf{Дальнейшие ресурсы}, colback=yellow!14!white, colframe=red!75!blue]
\begin{itemize}
	\item[+] Почему Открытый исходный код / Свободное программное обеспечение, Дэвид А. Уилер - статья об использовании Open Source Software / Free Software (OSS/FS).
	\item[+] Distrowatch, популярный сайт, который пытается отследить все доступные дистрибутивы Linux и имеет еженедельное освещение в новостях.
\end{itemize}
\end{tcolorbox}
\end{document}