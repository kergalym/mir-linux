\documentclass[12pt]{book}
\pagestyle{headings} % для более удобной навигации
\usepackage[utf8]{inputenc} % чтобы не глючило. Собирается через pdfTeX!!!
\usepackage[english,russian]{babel} % 2 языка
\usepackage[margin=2cm]{geometry} % стандартная ширина поля в книгах
\usepackage{setspace} % полуторный интервал
\usepackage{tcolorbox} % будет полезно
\usepackage{color} % будет полезно
\usepackage{droid} % вполне неплохо смотрится
\usepackage{graphicx}
\usepackage{pdfpages} % для вставки pdf при необходимости
\usepackage{hyperref}
\usepackage{indentfirst} % для отступов в основном тексте

%%% Поле для настройки отдельных %%%
%%  Элементов	ДО текста         %%
%\renewcommand{\abstractname}{О Книге}
%%%%%%%%%%%%%%%%%%%%%%%%%%%%%%%%%%%%
\begin{document}
%%% Поле для настройки отдельных %%%
%%  Элементов	по ходу           %%

%%% Оформление титульного листа %%%%
%   Титульный лист будет присое-  %%
%   диняться к файлу через pdf-   %%
%   shuffler, чтобы сработали     %%
%   все необходимые ссылки        %%
%%%%%%%%%%%%%%%%%%%%%%%%%%%%%%%%%%%%

%%%   Страница с информацией     %%%
\pagestyle{empty}
\textbf{Мир Linux}

\phantom{}
\textbf{Свен Вермален}

\phantom{}
Copyright © 2009-2013 Sven Vermeulen

\phantom{}
Перевод с английского: Галым Керимбеков\footnotetext{\href{kerimbekov.galym@yandex.ru}{$^{*}$kerimbekov.galym@yandex.ru}}$^{*}$

\phantom{}
Пост-верстка: Stas Bookman\footnotetext{\href{nodor4mint@yandex.ru}{$^{**}$nodor4mint@yandex.ru}}$^{**}$
%%% Аннотация %%%

\phantom{}

\phantom{}
\begin{tcolorbox}[colback=yellow!14!white]
\noindent Книга "Мир Linux"{} предлагает мягкое, но не лишенное технических аспектов (с точки зрения конечного пользователя) знакомство с операционной системой на примере дистрибутива Gentoo Linux. Книга не станет описывать историю ядра Linux или дистрибутивов, либо погружать в детали, менее интересные для пользователей.

\phantom{}
\noindent Онлайн-руководство Gentoo предлагает весьма подробный подход по ряду разделов и поэтому обязательно к чтению любым пользователем, желающим знать обо всех возможностях этой операционной системы. Несмотря на то, что издание "Мир Linux"{} и онлайн-руководство Gentoo определённо пересекаются между собой, книга ни в коем случае не призвана заменить последнее.

\phantom{}
\noindent "Мир Linux"{} попытается сосредоточиться на темах, о которых повседневные пользователи, вероятно, должны знать, чтобы продолжать работать с Gentoo Linux.

\phantom{}
\tcblower

\phantom{}
\noindent Версия, которую вы читаете в настоящее время, является v1.17 и была сгенерирована 18.02.2016 г. Доступны также версии ODT и ePUB.
\end{tcolorbox}

%%%  Оглавление  %%%
\tableofcontents %%%
%%%%%%%%%%%%%%%%%%%%

\newpage

%%%  Введение  %%%

\onehalfspacing % дальше по тексту -- полуторный интервал
%\pagestyle{headings} % С номерами страниц и заголовками

\section*{Введение}

В области настольных графических сред, Linux, предположительно, небольшой игрок (рыночные исследования оценивают долю на рынке приблизительно в 3\%). Однако наиболее вероятно, что вы знаете двух или более человек, использующих Linux, некоторые из них даже исключительно. Если принять это во внимание, то либо вы знакомы с предпочтениями ОС у более, чем ста человек, либо данную статистику стоит рассматривать с некоторым сомнением.

Тем не менее, 3\% - это все ещё много (Вы когда-нибудь думали о том, насколько много настольных систем? Я никогда не находил ответа на  этот вопрос). И если мы примем во внимание другие рынки (встраиваемые системы, серверы, сетевые устройства и прочее), доля Linux увеличится.

Но тем не менее, множество людей понятия не имеют, что такое Linux или как с ним работать. В этой книге я предлагаю техническое, краткое введение в операционную систему  с пользовательской точки зрения. Я не собираюсь погружаться в понятия, преимущества или недостатки Свободного Программного обеспечения (тем не менее, несколько абзацев не станут болезненными) и рассказывать об истории и развитии операционных систем Linux. Для ознакомления с большим числом ресурсов по этим темам, обратитесь к разделу «Дальнейшие ресурсы» в конце этой главы.

Чтобы стать полноценной книгой о Linux как об операционной системе, важно сообщить пользователю об операционных системах в целом. Linux очень модульный и открытый, и это подразумевает, что пользователю виден каждый компонент в системе. Без понимания структуры ОС пользователю было бы трудно осмыслить предназначение каждого модуля. Поэтому, я посвящаю весь раздел операционным системам.

Как только мы дойдём до задач операционной системы, я продолжу рассказ о реальных системах Linux: дистрибутивах.

В завершение, каждая глава в этой книге предложит ряд упражнений, которые можно попытаться решить. Вы не сможете найти в этой книге ответы по каждому вопросу. Предпочтительнее было бы взглянуть на упражнения, как на средство для дальнейшего продвижения и помощь в поиске (и нахождении) больше тем, связанных с Linux. В конце книги приведён список подсказок и/или ответов по вопросам.

\newpage

%%% Глава 1: Анатомия операционной системы %%%
\pagestyle{headings}
\chapter{Анатомия операционной системы}

Операционная система --- фактически стек программного обеспечения, каждый элемент, разработанный для определённой цели.

\begin{itemize}
	\item \textbf{Ядро системы}: управляет обменом данных между устройствами и программным обеспечением, системными ресурсами (такие как процессорное время, память, сеть...) и экранирует разработчика от сложности программирования устройства, предоставляя программисту интерфейс для управления аппаратными средствами.
	\item Системные библиотеки содержат программные методы для разработчиков, необходимые для написания программного обеспечения для операционной системы. Они включают в себя методы создания и манипулирования, обработки файлов, сетевого программирования, и т.д. Это - жизненно важная часть операционной системы, так как вы не можете (или не  должны) связываться с ядром напрямую: библиотека экранирует системного программиста от сложности программирования ядра.
	\item \textbf{Системные инструменты} собраны с использованием системных библиотек и позволяют администраторам следить за системой: управлять процессами, перемещаться по файловой системе, запускать другие приложения, настраивать сеть\ldots
	\item \textbf{Инструменты разработки} предоставляют средства для сборки нового программного обеспечения в (или для) системе. Несмотря на то, что это необязательная часть операционной системы, мне очень нравится упоминать их, так как с Gentoo их наличие является требованием (почему дело обстоит так, мы узнаем позже). Эти инструменты включают в себя компиляторы (переводят код в машинный), компоновщики (собирают машинный код и объединяют его в рабочий двоичный файл) и средства, значительно упрощающие процесс сборки.
\end{itemize}

Другие находящиеся в системе библиотеки улучшают опыт написания кода разработчиками, обеспечивая доступ к методам, которые уже написаны другими. Примеры таких библиотек включают в себя графические (для управления окнами) или научные библиотеки. Их наличие не обязательно в каждой системе, но если вы захотите запустить определённый инструмент, это потребует установку соответствующих библиотек. Поверх этих дополнительных библиотек вы обнаружите готовые к установке инструменты конечного пользователя (пакеты офисных программ, мультимедийные утилиты, графические среды\ldots).

\section{Ядро}

Как правило, ядро имеет четыре основные обязанности:
\begin{itemize}
	\item управление устройствами
	\item управление памятью
	\item управление процессами
	\item обработка системных вызовов
\end{itemize}

Первая из них называется управление устройствами. Компьютерная система имеет несколько соединённых с ней устройств: доступны не только ЦП и память, но также и диски (и дисковые контроллеры), сетевые платы, графические карты, звуковые карты... Так как каждое устройство работает по-разному, ядру необходимо знать, что может делать устройство, как адресовать и управлять каждым из устройств, обеспечивая корректное функционирование в  системе. Эта информация хранится в драйвере устройства: без такого драйвера ядро не знает об устройстве и поэтому не сможет управлять им.

Наряду с драйверами устройств, ядро также управляет обменом данных между ними: оно управляет доступом к совместно используемым компонентам так, чтобы все драйверы могли работать как единое целое. Весь обмен должен придерживаться строгих правил, а ядро подтверждать их соблюдение.

Компонент управления памятью управляет использованием памяти системой: оно отслеживает используемую и неиспользованную память, присваивает процессам,  требующим её, и гарантирует то, что они не смогут управлять данными друг друга. Чтобы осуществить это, ядро использует понятие адресов виртуальной памяти: адреса для одного процесса не являются действительными, и ядро отслеживает корректную картографию этих адресов. Также возможно, что в действительности данные отсутствуют в памяти, несмотря на то, что они присутствует для процесса: такие данные хранятся в области подкачки. Поскольку область подкачки намного медленнее реальной памяти, использование этого пространства должно быть ограничено данными, которые считываются нерегулярно.

Чтобы гарантировать, что каждый процесс получает достаточно процессорного времени, ядро отдаёт приоритеты процессам и даёт каждому из них определённое количество процессорного времени, прежде чем остановит процесс и передаст ЦП следующему. Управление процессами имеет дело не только с делегацией процессорного времени (называется планированием), но также и с полномочиями безопасности, информацией о владельце процесса, обменом данными между процессами и т.д.

Наконец, чтобы фактически работать над системой, ядро должно обеспечиваться средствами, необходимыми для системы и программиста, чтобы контролировать само себя, предоставлять или получать информацию, исходя из которой могут быть приняты новые решения. Используя системные вызовы программист может написать приложения, которые запрашивают ядро для получения информации или просят, чтобы оно выполнило определённую задачу (например, управлять некоторыми аппаратными средствами и возвращать результат). Конечно, такие вызовы должны быть безопасны в использовании,  чтобы вредоносные программисты не могли привести систему в нерабочее состояние  хорошо обработанным вызовом.

Операционная система, такая как Gentoo Linux, использует Linux в качестве ядра.

\section{Системные библиотеки}

Поскольку ядро само по себе мало на что способно, для выполнения задач оно должно инициироваться триггерами. Такие триггеры создаются приложениями, однако эти приложения  конечно должны знать, как устанавливать системные вызовы ядра. Поскольку каждое ядро имеет различный набор системных вызовов (очень специфичных по отношению к системе), программисты создали стандарты с которыми они могут работать. Каждая операционная система поддерживает эти стандарты, которые в дальнейшем переводятся в соответствующие системные вызовы для этой системы.

Стандарт в качестве примера - библиотека Cи, вероятно наиболее важная доступная системная библиотека. Эта библиотека делает доступными довольно первостепенные операции для программиста, такие как основная поддержка ввода/вывода, строковые подпрограммы обработки, математические методы, управление памятью и операции с файлами. С помощью этих функций программист может создать программное обеспечение, которое основывается на каждой операционной системе, поддерживающей библиотеку Cи. Эти методы далее  переводятся библиотекой Cи в определённые системные вызовы ядра (если они необходимы). Таким образом, у программиста нет необходимости знать о внутренностях ядра, он даже может написать программу (однократно), которая может быть собрана для множества платформ.

Не существует какой-либо единственной спецификации по тому, какова системная библиотека. Автор этой книги полагает, что системные библиотеки — это любые библиотеки, являющиеся частью стандартной, минимальной инсталляции операционной системы. Также системные библиотеки для одной операционной системы (и даже дистрибутив) могут отличаться от аналогичных библиотек для другой. У большинства дистрибутивов Linux имеются те же системные библиотеки, что неудивительно, так как все они могут выполнять одно то же программное обеспечение, если конечно оно собрано для этих библиотек. Некоторые дистрибутивы просто не отмечают одну библиотеку как часть стандартной, минимальной инсталляции, в отличие от других.

Самая распространённая системная библиотека для систем Linux - GNU Cи, также известная как glibc.

\section{Системные инструменты}

Так же как и с системными библиотеками, не существует единой спецификации для системных инструментов. Тем не менее, в отличие от системных библиотек они достаточно прозрачны для конечного пользователя. Поэтому почти все дистрибутивы Linux используют те же системные инструменты или подобные инструменты с  аналогичными функциями, но реализованные иначе.

Но что такое системные инструменты? Ну, вы все ещё не можете управлять своей системой с помощью ядра и некоторых библиотек программирования. Для этого вам необходим доступ к командам для ввода в интерпретированную и запущенную систему. Эти команды выполняют простейшие задачи, такие как навигация по файлам (переход в каталог, создание/удаление файлов, получение файла листингов...), управление информацией (текстовый поиск, сжатие, перечисление различий между файлами...), управление процессом (запуск  новых процессов, получение листинга процессов, завершение рабочих процессов...), задачи, связанные с полномочиями (изменение владельца файлов, изменение идентификаторов пользователей, обновление разрешений файлов...) и др.

Если вы не знаете, как иметь дело со всем этим материалом, то не имеете навыков работы с системой. Некоторые операционные системы скрывают эти задачи за усложнёнными средствами, другие имеют простые, но отдельные для каждой задачи и объединяют мощь всех этих инструментов. Unix (и Linux) является одним из последних. Системы Linux для большинства этих задач обычно имеют набор системных утилит GNU (GNU Core Utilities).

\section{Средства разработки}

С упомянутыми тремя компонентами у вас есть запущенная, рабочая  система. Возможно вы не сможете сделать всё, что захотите, однако можно обновить систему до того состояния, прежде чем она станет выполнять желаемое. Каким образом? Устанавливая дополнительные инструменты и библиотеки до тех пор, пока у вас не будет своей функционирующей системы.

Эти дополнительные инструменты и библиотеки, несомненно, написаны программистами, а это значит, что они должны быть способны собрать свой код так, чтобы он работал в системе. Некоторые системы, такие как Gentoo Linux, даже собирают их вместо того, чтобы полагаться на исходный код, предварительно собранный другими людьми. Чтобы собрать эти инструменты, необходим исходный код каждого из них,  а также инструментов, необходимых для преобразования исходного кода в исполняемые файлы.

Называются они «Инструментарий разработки» (tool chain): набор связанных между собой инструментов, необходимых для разработки рабочего приложения. Общий набор инструментов разработки состоит из текстового редактора (для написания кода), компилятора (для преобразования кода в специфичный для машины язык), компоновщика (для объединения машинного кода нескольких исходных текстов - включая предварительно собранные "совместно используемые" (shared) библиотеки в единый исполняемый файл) и библиотек (их я просто упоминал как "совместно используемые").

Этот инструментарий имеет наибольшее значение для разработчика; это — жизненно важное средство разработки, но не единственное. К примеру, разработчики графических приложений обычно нуждаются в инструментах для создания графики, или даже инструментах, связанных с мультимедией, для добавления звуковых эффектов в их программу. Средство разработки - общее существительное для инструмента, необходимого разработчику для создания чего-либо, но несущественного для операционной системы пользователя среднего уровня, который не является разработчиком.

Наиболее известные средства разработки также предоставляются Фондом GNU, а именно, набор компиляторов GNU (GNU Compiler Collection), также известный как gcc.

\section{Инструменты, предоставляемые конечному пользователю}

Как только разработчик закончит создание своего продукта, у вас появится инструмент конечного пользователя с сопроводительными библиотеками (которые могли бы потребоваться другим инструментам, который представляют собой модифицированные сборки на основе этого продукта). Инструменты, делающие систему уникальной для пользователя, представляют собой то, что он хочет сделать с ней. Несмотря на невостребованность самой  операционной системой, они необходимы конечному пользователю и поэтому очень важны для его системы.

Большинство операционных систем не устанавливает все инструменты или большинство их них по причине того, что их слишком много чтобы выбрать. Некоторые операционные системы даже не обеспечивают пользователя средствами их установки, но полагаются на изобретательность программиста, на создание им установщика, который интегрирует инструмент в систему. Иные  системы поставляют с собой небольшое, но значительное подмножество инструментов конечного пользователя, чтобы их пользователи могли быстро обновить свои системы в любом виде, каком они пожелают, не требуя долгого и изнурительного поиска через Интернет (или ещё хуже, магазин аппаратного или программного обеспечения)  программ, в которых они нуждаются.

Примеры инструментов конечного пользователя известны как пакеты офисных программ, инструменты графического дизайна, мультимедийные проигрыватели, коммуникационное программное обеспечение, интернет-браузеры.

\section{Хорошо, что такое этот GNU?}

Проект GNU - усилие нескольких программистов и разработчиков по созданию свободной,  Unix-подобной операционной системы. GNU является рекурсивным акронимом, означающим, что GNU не Unix, поскольку она подобна ему, но не содержит его кода и (и остаётся) является свободной. Фонд GNU, юридическое лицо, стоящее за проектом GNU, видит свободу как нечто большее, чем просто свободу в финансовом значении: программное обеспечение должно быть свободным для использования в любых целях вообще,  для изучения,  изменения исходного кода ПО и его поведения, копирования и свободного распространения изменений, внесённых вами.

Данная идея свободного программного обеспечения - благородная мысль, активная в умах многих программистов: следовательно, множество тайтлов программного обеспечения находится в свободном доступе. Программа обычно сопровождается лицензией, которая объясняет, что вы можете и не можете делать с ней (также известное как "Лицензионное соглашение для конечного пользователя" - EULA). Свободное программное обеспечение также имеет подобную лицензию - в отличие от EULA, вместо того, чтобы запрещать она фактически многое позволяет. Пример такой лицензии - GPL - Универсальная общедоступная лицензия GNU.

\section{Linux в качестве ядра операционной системы}

Когда мы смотрим на операционную систему Linux, её основным компонентом является ядро. То самое, которое используют все подобные системы - ядро Linux или просто Linux. Да, правильно, систему Linux называют по имени ядра.

Теперь несмотря на то, что все операционные системы Linux используют одноимённое ядро, многие из них используют различные адаптированные версии. Это вызвано тем, что  его разработка имеет несколько ответвлений. Самым важным я назову ванильное ядро. Основное ядро разработки, где  продолжает работать большинство его разработчиков; любое другое основывается на нём. Другие ядра предоставляют функции, которые ванильное ядро  ещё не имеет (или отказалось в пользу других); тем не менее, они полностью с ним совместимы.

Ядро Linux впервые увидело  свет в 1991 году, разрабатывается (и все ещё актуально) Линусом Торвальдсом. Развивалось оно стремительно (версия 1.0.0 уже в 1994 г.) как в размере (версия 1.0.0 имела более чем 175000 строк кода), так и в популярности. За эти годы его модель разработки осталась прежней: существуют немного крупных игроков в разработке, которые решают, что входит и что остаётся вне кода ядра, но большинство вкладов происходит от десятки сотен волонтёров (усвой вклад в ядро версии 2.6.21 несло более чем  800 человек).

Последняя версия ядра на момент написания этой книги - 3.10.7. Первые два числа играют роль основной версии, третье число - вспомогательная версия (главным образом, обозначающее выпуски с исправлениями ошибок). Иногда добавляется четвёртое число, если необходимо одноразовое исправление ошибки. При разработке ядра Linux обычно постепенно увеличиваются основные числа (большую часть времени второе число), обозначающие выпуски с функциональными улучшениями: с каждым инкрементом пользователи (и разработчики) узнают, что таким образом, ядро получает новые функции.

Как только выпускается новая версия ядра Linux, оно не распространяется всем его пользователям. Нет, это уже та область, где играют роль дистрибутивы\ldots

\section{Операционные системы Linux: дистрибутивы}

Если бы пользователь захотел установить систему Linux без дополнительной справки, ему пришлось бы самостоятельно собрать ядро, компоненты операционной системы (библиотеки, конечные инструменты...) и отслеживать изменения в мире свободного программного обеспечения (такие как выход новых версий или исправлений безопасности). И несмотря на то, что всё это совершенно возможно (обратитесь к проекту Linux From Scratch), большинство пользователей пожелало бы что-нибудь более... удобное для них.

Откройте для себя дистрибутивы. Проект дистрибуции (такой как Проект Gentoo) ответственен за расширение системы Linux (дистрибутива) для конечного пользователя, по сути точка контакта для её установки.

Проекты дистрибуции делают выбор относительно программного обеспечения:

\begin{itemize}
	\item[-] Каким образом пользователи должны установить операционную систему?
\end{itemize}

Возможно, пользователи поощряются к выполнению как можно больше шагов в  процессе установки ("дистрибутив" Linux From Scratch, вероятно имеет самый интенсивный процесс установки). Наиболее противоположным этому является загрузочный CD или USB образ системы, которая даже не требует какой-либо настройки или установки: она просто загружает среду и вы можете начать использовать такой дистрибутив.

\begin{itemize}
	\item[-] Какие имеются варианты установки (CD, DVD, сеть, Интернет\ldots?)
\end{itemize}

Большинство дистрибутивов Linux предлагает вариант установки с CD/DVD, поскольку это самый популярный способ для получения программного обеспечения. Однако существуют множество других вариантов. Можно установить дистрибутив по сети, используя начальную сетевую загрузку (популярный подход в корпоративной среде, так как он делает возможной установку без сопровождения), или из другой операционной системы.

\begin{itemize}
	\item[-] Какое программное обеспечение должно быть доступно пользователю?
\end{itemize}

Популярные настольные дистрибутивы предлагают конечным пользователям широкий диапазон программного обеспечения. Это позволяет дистрибутиву стать общепризнанным, поскольку он соответствует потребностям многих пользователей. Однако существуют более усовершенствованные дистрибутивы, которые сконцентрированы на определённый рынок (абонентские установки для мультимедийных представлений, брандмауэры и управление сетью домашние устройства автоматизации...) и конечно эти дистрибутивы предлагают пользователям различные тайтлы программ.

\begin{itemize}
	\item[-] Каким образом собрано доступное программное обеспечение (определённая система, функции\ldots)?
\end{itemize}

Если дистрибутиву требуется, чтобы программное обеспечение работало на наибольшем количестве типов процессоров (Pentium, i7, Athlon, Xeon, Itanium\ldots), необходимо  собрать программное обеспечение для универсальной платформы (скажем i686), а не для определённой (Itanium). Конечно, это означает, что программа не станет использовать все функции, которые обеспечивают новые процессоры, но в действительности оно все ещё будет работает на множестве систем.

То же самое справедливо и для функций, поддерживаемых определёнными тайтлами программ. Некоторые из них предлагают дополнительную поддержку ipv6, ssl, truetype fonts... Однако, при необходимости,  вы должны скомпилировать её в программе. Дистрибутивы, предлагающие программное обеспечение в двоичном формате (этим и занимается большинство из них) должны сделать этот выбор за своих пользователей.  Чаще всего, они пытаются предложить поддержку как можно большего количества функций, однако далеко не все конечные пользователи нуждались бы этом или даже желали бы.

\begin{itemize}
	\item[-] Так ли важна интернационализация программного обеспечения?
\end{itemize}

Некоторые дистрибутивы предназначаются для определённых групп пользователей, в зависимости от языка и географии. Существуют  полностью локализованные для определённой группы (скажем для "пользователей, говорящих на бельгийско-нидерландском языке" или "канадских франкоговорящих пользователей"), но также и пытающиеся предложить локализацию для наибольшего количества групп.

\begin{itemize}
	\item[-] Как пользователи должны обновлять и обслуживать свою систему?
\end{itemize}

Множество дистрибутивов предлагают автоматизированный, но не у всех в наличии живой процесс обновления программного обеспечения (например, когда-то установленная вами инсталляция постепенно растёт и становится последней версией дистрибутива без каких-либо определённых действий). Некоторые из них даже требуют загрузку с последнего установочного CD и выполнение шагов по обновлению.

\begin{flushright} \emph{Каким образом пользователь настраивал бы свою систему?}
\end{flushright}

Если вы --- пользователь графического варианта Linux, то определённо не хотите слышать о редактировании конфигурационных файлов или действиях, связанных с командной строки. Таким образом наиболее вероятно, вы станете искать дистрибутивы, предлагающие полный графический интерфейс для настройки системы. Тем не менее некоторым пользователям в самом деле нравится идея редактирования таких файлов напрямую, так как она предлагает наибольшую гибкость (но также и самую высокую кривую обучения) и дистрибутивы чаще всего предлагают эти предпочтения. Некоторые дистрибутивы даже не позволят обновить их напрямую, так как они  (пере)создаются без вмешательства, так или иначе перезаписывается всё, что вы изменяли.

\begin{flushright} \emph{Какова целевая группа пользователей дистрибутива?}
\end{flushright}

Большинство настольных дистрибутивов предназначены для домашних/офисных пользователей, но существуют дистрибутивы для детей или учёных. Некоторые созданы для разработчиков, а другие для людей старшего возраста, слабовидящих и не имеющих доступ к Интернету.

\begin{flushright} \emph{Какую политику дистрибутив использует в своём программном обеспечении?}
\end{flushright}

Организации, такие как FSF, имеют собственную философию касательно того, на что мир (программного обеспечения) должен быть похожим. Множество дистрибутивов предлагают способ реализовать это. Например, некоторые из них допускают лишь программное обеспечение, лицензируемое в соответствии с утверждённой FSF лицензией. Другие дистрибутивы позволяют пользователям использовать несвободное. Есть и реализующие более высокую философию безопасности, предлагая наиболее защищённый подход к операционным системам.

\begin{flushright} \emph{Дистрибутив должен быть в свободном доступе?}
\end{flushright}

Конечно, деньги --- нередко основная причина принятия решений. Не все дистрибутивы свободно загружаемы/доступны в Интернете, несмотря на то, что большинство из них таково. Но даже когда дистрибутив находится в свободном доступе, все ещё есть необходимость получить коммерческую поддержку, даже только для обновлений безопасности дистрибутива.

Вы обнаружите несколько дистрибутивов; каждый из это проектов отвечает на вопросы, которые немного отличаются от других. Следовательно, выбор правильного дистрибутива часто является задачей, в которой вы должны ответить на многие вопросы, прежде чем найти подходящий.

Конечно, когда вы только начинаете работать с Linux, у вас скорее всего ещё нет твёрдого мнения касательно этих вопросов. Это хорошо, так как если вы хотите  использовать Linux, то должны начать с дистрибутива, который обеспечивает наилучшую поддержку. Расспросите кого-нибудь рядом, возможно у вас есть друзья, которые могли бы помочь. И будем честны, что может быть лучше персональной поддержки?

\section{Что такое дистрибутив?}

Дистрибутив --- это набор программного обеспечения (названного пакетами), объединённого вместе в единый набор, который создаёт полностью функциональную среду. Пакеты содержат тайтлы программного обеспечения (сборка других проектов) и возможно содержат исправления (обновления), специфичные для дистрибутива для лучшей интеграции пакетов или гармоничности со всей средой. Эти пакеты обычно содержат не только копию выпусков, созданных другими проектами программного обеспечения, но и большую логику с целью приспособить программное обеспечение к  философии дистрибутива.

Возьмите к примеру KDE. KDE - (графическая) настольная среда, связывающая вместе несколько десятков инструментов меньшего размера. Некоторые дистрибутивы предлагают своим пользователям оригинальную инсталляцию KDE, другие немного изменяют ее для достижения индивидуального оформления по умолчанию и т.п.

Другим примером был бы MPlayer, мультимедийный проигрыватель, наиболее известный широкой поддержкой различных видеоформатов. Однако, если вы хотите просмотреть видеофайлы Windows Media (WMV), в этом случае потребуется встроенная поддержка (несвободных) win32 кодеков. Некоторые дистрибутивы предоставляют её для MPlayer, другие нет. Gentoo Linux позволяет выбирать, хотите вы её или нет.

\section{Что обеспечивает дистрибутив?}

Если вы хотите использовать дистрибутив, то можете (но не обязаны) использовать инструменты, созданные проектом дистрибутива для упрощения нескольких задач:

\begin{itemize}
	\item для установки дистрибутива, можно использовать одну или более программ установки, обеспечиваемых проектом
	\item чтобы установить дополнительные пакеты в системе, можно использовать один или несколько инструментов управления программным обеспечением, обеспечиваемых проектом
	\item для настройки системы можно использовать один или несколько инструментов настройки, обеспечиваемых проектом.
\end{itemize}

Я не могу не подчеркнуть достаточную важность термина. Вы не обязаны использовать программу установки дистрибутива (всегда можно установить дистрибутив по-другому), и не обязаны устанавливать программы, используя инструменты управления программным обеспечением (можно собирать и устанавливать вручную), и не обязаны настраивать систему инструментами  для настройки (вы можете вручную редактировать конфигурационные файлы различных приложений).

Почему в таком случае дистрибутив перекладывает все эти усилия на инструменты? Причина заключается в том, что они намного упрощают использование системы для пользователя.  Возьмите в качестве примера установку программ. Если вы не используете инструмент управления программным обеспечением, то должны собрать программу самостоятельно (что может отличаться в зависимости от программы, которую вы хотите собрать), отслеживать обновления (исправления ошибок и безопасности), удостовериться в том, что установили все зависимое программное обеспечение (программное обеспечение, которое зависит от другого, что, в свою очередь, зависит от библиотеки a, b и c...), и отслеживать установленные файлы, чтобы система не переполнилась хламом.

Другое основное дополнение, которое обеспечивают дистрибутивы — это пакеты программного обеспечения. Пакет содержит программный тайтл (например браузер Mozilla Firefox) с дополнительной информацией (такой как описание программного тайтла, информация о категории, зависимостях и библиотеках...) и логику (как установить программное обеспечение, как активировать определённые модули, которые оно обеспечивает, как создать запись меню в графических средах, как его собрать, если он ещё не собран...). Это может отразиться на сложности пакета, что является одной из причин того, почему дистрибутивы обычно не могут выпустить новый пакет в день выпуска версии программы.

Однако, большая часть информации и логики для исправлений безопасности остаётся такой же, выпуск исправлений безопасности для программного обеспечения обычно приводит к быстрому выпуску проектом дистрибутива исправлений безопасности к пакету с этим ПО.

Проект дистрибутива обеспечивает следующие элементы поддержки наряду с программным обеспечением из которого состоит дистрибутив:

\begin{itemize}
	\item документация по дистрибутиву
	\item инфраструктура, откуда Вы можете загрузить дистрибутив и его документацию
	\item ежедневные обновления пакетов для нового программного обеспечения
	\item ежедневные обновления безопасности
поддержка по дистрибутиву (которая может быть в виде форумов, почтовой, телефонной или даже коммерческой договорной),
 \item \ldots
\end{itemize}

Отныне проект дистрибутива — нечто большее, чем всё это. Разработчики могут сотрудничать, объединяя все пакеты в единый проект, чтобы создать  систему, которая расширяет ряд операционных систем "коммерческого сорта". Для достижения этой цели большинство проектов дистрибутива имеет подразделения для связей с общественностью, пользовательских отношений, отношений с разработчиками, управления версиями, документации и переводов и т.д.

\section{Что такое архитектура?}

Я ещё не упоминал об архитектурах, тем не менее они важны. Позвольте мне для начала определить понятие набора инструкций.

Набор инструкций ЦП - это набор команд, который понимает определённый процессор. Эти команды выполняют множество функций, таких как арифметические функции, операции памяти и управление потоком. Программы могут быть написаны с их использованием, но обычно программисты применяют высокоуровневый язык программирования, поскольку  программа, написанная на этом языке (названном ассемблером упомянутого ЦП), может запускаться только на этом ЦП. Таким образом, ассемблер настолько низкоуровневый, что написать инструмент с помощью него не совсем просто. Инструменты, использующие ассемблер — это компиляторы (которые переводят высокоуровневый язык программирования в ассемблер), загрузчики (которые загружают операционную систему в память), и некоторые базовые компоненты операционных систем (ядро Linux имеет некоторый ассемблерный код).

Теперь у каждого типа ЦП имеется различный набор инструкций. Intel IV Pentium имеет набор инструкций, отличающийся от набора Intel Pentium III; Sun UltraSPARC III  имеет набор инструкций, отличающийся от набора Sun UltraSPARC III. Однако они очень схожи. Это вызвано тем, что они находятся в том же семействе. Центральные процессоры того же семейства понимают определённый набор инструкций. Программные инструменты, созданные для одной системы команд, работают на всех центральных процессорах того же семейства, но не могут использовать в своих интересах весь набор инструкций ЦП, на котором они работают.

Семейства центральных процессоров сгруппированы по архитектурам. Архитектура - глобальная переменная и представляет понятие всей системы; она описывает, как получает доступ к дискам, как обрабатывается память, как определяется процесс начальной загрузки. Она определяет большие, концептуальные различия между системой. Например, диапазон систем, совместимый с Intel, сгруппирован в архитектуре x86; если вы загружаете такую систему, её процесс начальной загрузки запускается с BIOS (Базовая система ввода-вывода). Системы, совместимые со Sparc Sun сгруппированы в архитектуре sparc; если Вы загружаете такую систему, процесс начальной загрузки запускается с PROM.

Архитектура важна, так как дистрибутивы Linux зачастую поддерживают несколько архитектур, и вы определённо должны знать, какую архитектуру  ваша система использует. Это по всей вероятности x86 или amd64 (оба довольно эквивалентны), однако необходимо понять, что существует также и другая архитектура. Вы найдёте инструменты, которые не поддерживаются для вашей архитектуры даже при том, что они доступны для вашего дистрибутива, или некоторые пакеты могут иметь последнюю версию в наличии на одной архитектуре, но не на других

\section{Мифы, окружающие Linux}

Linux часто расхваливается в СМИ - иногда этому есть объяснение, однако большую часть времени его нет. И хотя я ранее обсуждал, чем является Linux, напомню вкратце:

\begin{quote}
\emph{Linux - общее обозначение, относящееся к операционной системе Linux, набору инструментов, работающих под управлением ядра Linux и большую часть времени предлагаемых через проект дистрибутива.}
\end{quote}

Конечно, зачастую это не вносит ясность для пользователей, незнакомых с миром вне Microsoft Windows. Несмотря на то, что лучший способ узнать, что такое Linux, это использовать его, мне кажется важным разоблачить некоторые мифы, прежде чем продолжить остальную часть книги.

\section*{Мифы}

Миф — история, которая популярна, но не верна.  Мифы, окружающие Linux будут существовать всегда. Следующие несколько разделов попытаются предложить мои идеи касательно развенчанию большинства из них...

\subsection{Linux трудно установить}

Кто-то всегда может указать на дистрибутив, который трудно установить. "Дистрибутив" Linux From Scratch - фактически документ, объясняющий весь процесс установки дистрибутива Linux путём сборки компиляторов, программного обеспечения, размещением файлов, и т.д. Да, это тяжело и даже могло быть сложным, если бы документация не была актуальной.

Тем не менее, множество дистрибутивов (даже большинство из них) просты в установке. Они предлагают тот же подход к установке, как и другие операционные системы (включая Microsoft Windows), вместе с онлайн-справкой (экранная справка) и оффлайн-справкой (руководство по установке). Некоторые дистрибутивы могут даже быть установлены всего двумя или тремя вопросами, вы можете даже использовать Linux без необходимости  устанавливать его вообще.

\subsection{Для Linux нет поддержки }

Были дни, когда для Linux не существовало коммерческой поддержки, но это было в прошлом веке. Теперь Вы можете получить операционную систему Linux от крупных поставщиков программного обеспечения, таких как Novell или RedHat (с поддержкой), или использовать свободно загружаемый дистрибутив Linux и получить контракт с компанией, которая предлагает поддержку этого дистрибутива.

Все дистрибутивы также предлагают превосходную бесплатную поддержку (это то, о чем я расскажу в следующих нескольких главах) и у многих из них в наличии активная последующая обработка и анализ безопасности, приводящие к быстрым исправлениям, как только уязвимость находят или сообщают о ней. Часто нет никакой потребности в получении коммерческой поддержки для пользователя настольных систем, поскольку каналы поддержки в свободном доступе предлагают главное преимущество по сравнению с некоторыми другими собственническими операционными системами.

\subsection{Linux - свободное программное обеспечение, таким образом, дыры в системе безопасности находятся легко}

Так как это свободное программное обеспечение, «дырам» в системе безопасности намного труднее остаться в исходном коде. Существует слишком много глаз, наблюдающих за исходным кодом, а у многих проектов свободного программного обеспечения имеется очень активное сообщество разработчиков, которое проверяет и перепроверяет изменения исходного кода множество раз, прежде чем они будут предложены конечному пользователю.

\subsection{Linux не имеет графику}

Ядро Linux не является графическим ядром, однако инструменты, функционирующие ниже ядра, могут быть графическими. Даже больше, большинство дистрибутивов предлагает полный графический интерфейс для каждого возможного аспекта операционной системы: она загружается с графикой, вы работаете графическим методом, устанавливаете программное обеспечение графическим образом и даже диагностируете проблемы, используя графический подход. Несмотря на то, что вы можете работать исключительно с командной строкой, большинство дистрибутивов сконцентрировано на графической среде.

Эта книга не является хорошим примером относительно этого мифа, поскольку она фокусируется на командной строке. Однако это лишь из-за личных предпочтений автора.

\subsection{Я не могу запустить свою программу в Linux}

Для многих тайтлов Microsoft Windows это действительно так. Но почти наверняка существует программное обеспечение, доступное в Linux и предлагающее те же функции, что и то, к которому вы обращаетесь. Некоторые программы даже доступны для Linux: популярные браузеры Firefox и Chrome - два примера, пакет офисных программ в свободном доступе, OpenOffice.org — ещё один пример.

Существуют также эмуляторы и библиотеки, которые предлагают интерфейс, позволяющий приложениям Microsoft Windows работать в Linux. Тем не менее, я не рекомендую использовать данное программное обеспечение для каждого возможного тайтла. Это самое последнее средство в том случае, когда вам определённо требуется конкретный тайтл, но при этом вы уже выполняете большинство работы в Linux.

\subsection{Linux безопасен}

Это также миф. Linux не более безопасен, чем Microsoft Windows или Apple Mac OS X. Безопасность — это больше, чем сумма всех уязвимостей в программном обеспечении. Она основана на компетентности пользователя, администратора, конфигурации системы и др.

Linux может быть очень безопасным: существуют дистрибутивы,  фокусирующиеся на интенсивной безопасности посредством дополнительных параметров настройки, конфигураций ядра, выбора программного обеспечения и прочего. Но вы не нуждаетесь в таком дистрибутиве, если хотите иметь безопасную систему Linux. Предпочтительнее всего прочесть документацию по безопасности дистрибутива, и удостовериться в том, что вы регулярно обновляете свою систему, не запускаете программное обеспечение, в котором не нуждаетесь или не посещаете сайты, законность которых вам неизвестна.

\subsection{Linux слишком фрагментирован, чтобы когда-либо стать более крупным игроком}

Множество групп именуют Linux как фрагментируемый по причине множества дистрибутивов. Однако пользователь одного дистрибутива может легко работать с пользователями других дистрибутивов (здесь нет никакой проблемы). Пользователь одного дистрибутива может также помочь пользователям других дистрибутивов, так как их программное обеспечение - все ещё то же самое (здесь также нет никакой проблемы). Даже больше, программное обеспечение, создаваемое на одном дистрибутиве, прекрасно работает на другом дистрибутиве (здесь тоже нет никакой проблемы). Широко распространённая доступность дистрибутивов - это сила, а не слабость, поскольку она предлагает больше выбора (а также больше знаний и опыта) конечному пользователю.

Возможно, люди ссылаются на различные существующие деревья ядра Linux. Тем не менее, все эти деревья основаны на том же ведущем ядре (часто называемом как "ванильное") и каждый раз, когда ведущее ядро производит новую версию, эти деревья обновляют свой собственный код, поэтому ответвления никогда не отстают. Дополнительные деревья, которые существуют из-за целей разработки (дополнительные патчи для неподдерживаемых аппаратных средств, прежде чем они будут объединены с ведущим ядром, дополнительными патчами для определённых решений для виртуализации, которые в противном случае становятся несовместимыми или не могут быть объединены из-за проблем лицензии, дополнительные патчи, которые слишком навязчивы и требуют времени, прежде чем они будут стабилизированы, и т.д.).

Возможно, люди ссылаются на различные графические среды (такие как KDE и GNOME). Тем не менее, они умалчивают о функциональной совместимости между этими графическими средами (вы можете запустить приложения KDE в GNOME и наоборот), о стандартах, которые это разнообразие создают (стандарты работающие с форматами файлов, записями меню, связывании объектов и прочее), и прочее.

Управляемая фрагментация - это то, что предлагает Linux (и свободное программное обеспечение в целом). Управляемая, так как соответствует открытым стандартам и свободным спецификациям, которые хорошо документированы и которых придерживается все программы. Фрагментированная, так как сообщество хочет предложить  конечным пользователям больше выбора.

\subsection{Linux - альтернатива Microsoft Windows}

Linux - не альтернатива, это другая операционная система. Существует различие между значениями. Альтернативы пытаются предложить ту же функциональность и интерфейс, используя различные средства. Linux - иная операционная система, так как не стремится предложить ту же функциональность или интерфейс Microsoft Windows.

\subsection{Linux — это противовес Microsoft}

Это не так, поскольку люди, у которых есть определённые чувства к Microsoft, часто используют и Linux. Linux пытается быть, ни чем иным, как системой, полностью взаимодействующей с любой другой. Проекты программного обеспечения совершенно определённо желают, чтобы их программное обеспечение работало на любой операционной системе, не только на Microsoft Windows или Linux.

\subsection{Слабые места}

И все же не вся информация распространяется вокруг мифов. Некоторые из них - реальные слабые места, над которыми в Linux все ещё нужно продолжать работать.

\subsubsection{Поддержка игр прогрессирует медленно}

Это правда. Несмотря на то, что вокруг существует множество игр среди свободного программного обеспечения, большинство их них разработано исключительно для Microsoft Windows, и не все игры могут быть запущены в Linux с использованием эмуляторов или библиотек, таких как WINE (но к счастью, множество). Не так уж просто попросить разработчиков игр разработать их для Linux, поскольку большинство из них концентрирует свои усилия на библиотеках (такие как DirectX), доступных только для Microsoft Windows.

Тем не менее, в этой области недавно произошли улучшения. Valve выпустила Steam для Linux, обеспечивая игровой процесс в настольном Linux. Это дало большой сдвиг "играм на Linux".

Однако также появляется и другая тенденция: все больше игр  выпускается только для консолей, отбрасывая среду ПК. Я лично не знаю, как игры разовьются в будущем, но думаю, что реальные экшн-игры будут фокусироваться больше на игровые приставки.

\subsubsection{Новейшие аппаратные средства принимаются в Linux не сразу}

Если поставщик аппаратных средств не предлагает драйверы для Linux, то действительно потребуется некоторое время, прежде чем поддержка аппаратных средств будет обеспечена в ядре Linux. Однако это процесс, охватывающий не многие годы, а скорее месяцы. Возможность состоит в том, что совершенно новая видео- или звуковая карта будет поддерживаться уже в течение 3 - 6 месяцев после выпуска.

То же является истиной и для карт беспроводной сети. Принимая во внимание то, что ранее это было слабостью, поддержка карт беспроводной сети отныне хорошо скоординирована в сообществе. Основная причина этого состоит в том, что большинство поставщиков теперь официально поддерживает свой беспроводной чипсет для Linux, предлагая драйверы и документацию.

\newpage
{\color{white}\section{Упражнения}}
\begin{tcolorbox}[title=\textbf{Упражнения}, colback=yellow!14!white, colframe=red!75!white]
\begin{enumerate}
	\item Создайте список дистрибутивов Linux, о которых Вы услышали, и проверьте каждый из них, как они выполняют свои задачи в сферах, которые Вы находите важными (например, доступность документации, переводов, поддержки определённых аппаратных средств, мультимедиа\ldots).
	\item Перечислите 7 архитектур ЦП.
	\item Почему новые выпуски ядра не распространяются конечному пользователю незамедлительно? Какую роль дистрибутивы играют в этом процессе?
\end{enumerate}
\end{tcolorbox}

\phantom{}
\begin{tcolorbox}[title=\textbf{Дальнейшие ресурсы}, colback=yellow!14!white, colframe=red!75!blue]
\begin{itemize}
	\item[+] Почему Открытый исходный код / Свободное программное обеспечение, Дэвид А. Уилер - статья об использовании Open Source Software / Free Software (OSS/FS).
	\item[+] Distrowatch, популярный сайт, который пытается отследить все доступные дистрибутивы Linux и имеет еженедельное освещение в новостях.
\end{itemize}
\end{tcolorbox}

\newpage

%%% Глава 2: Как Свободное программное обеспечение влияет на Linux? %%%

\chapter{Как Свободное программное обеспечение влияет на Linux?}

\section*{Введение}

ОС Linux стала все более и более популярной главным образом из-за свободы, которую она позволяет (и конечно же низкая цена или стоимость нулевого сбора всей операционной системы). В этой главе мы увидим, как эти свободы претворяются в жизнь, как они защищены и поддерживаются.

Мы также взглянем на модель разработки, используемую проектами свободного программного обеспечения, так как это - главный результат упомянутых свобод, того, что часто делает такие проекты более интересными, чем коммерческие проекты программного обеспечения с закрытым исходным кодом. Модель разработки - также одно из главных сильных мест свободного программного обеспечения.

\section{Свободное программное обеспечение}

Если мы отдалимся от всех технических аспектов, то заметим, что Linux отличается от коммерческого программного обеспечения с закрытым исходным кодом в одном важном аспекте: лицензирование. Лицензирование --- это то, что управляет свободным программным обеспечением\ldots

\section{Что такое лицензии на программное обеспечение?}

Программное обеспечение --- чья-то интеллектуальная собственность. Интеллектуальная собственность --- тяжёлое слово, которое не должно быть интерпретировано ни к чему иному, кроме как к результату некоторого усилия по созданию того, что не является простой копией. Если вы пишете что-то, получившийся текст — это ваша интеллектуальная собственность (если вы не скопировали его где-нибудь).

Интеллектуальная собственность защищена законом. Авторское право защищает  интеллектуальную собственность, запрещая другим копировать, адаптировать, воспроизводить и/или перераспространять вашу ''вещь'' без вашего согласия. Тем не менее, имейте в виду, что не каждая интеллектуальная собственность защищена авторским правом, а оно отличается от страны к стране. Примером интеллектуальной собственности,  не защищённой авторским правом, является математический метод: даже при том, что изобретатель метода должен был потратить многие годы на его обдумывание, его метод не защищён (но если он написал текст об этом методе, то сам текст). Авторское право присваивается автоматически: это ничего не стоит Вам и принято в широком аспекте.

Другая защита --- патент. Патенты предоставляются (или должны) новым изобретениям,  не известным общественности во время запроса патента. Они часто используются для защиты интеллектуальной собственности, которая не защищена авторским правом: методов создания материала (в том числе медицинского содержания). К сожалению, отрасль часто злоупотребляет патентами намного больше, когда у них есть патент с широким спектром действия: он покрывает слишком многое, позволяя компании вынудить других не использовать метод, который они фактически имеют право использовать. Кроме того,  запрос и получение патента очень дорогостоящие, и только более крупные компании имеют возможность получить (и защитить) несколько патентов. У меньших компаний или людей нет средств получить патент, не говоря уже о собственной защите в суде лишь потому, что они, возможно, использовали метод, который описан в одном или более патентах.

Я использую слово «злоупотребление», так как компании часто получают патенты для методов, которые широко используются или настолько несерьёзны, что вы задались бы вопросом,  что за патентное бюро предоставило те патенты (запросы патентов проверяются или должны быть проверены на свою законность, прежде чем их предоставят).

Я воздержусь от развития этой (политически неоднозначной) темы в дальнейшем и перейду к лицензиям на программное обеспечение. Лицензия - это контракт между Вами - пользователем и владельцем авторского права на программное обеспечение. Она указывает, что вы можете и не можете делать. Любое  нелицензируемое программное обеспечение, полностью защищено авторским правом, что означает, что вы не можете иметь его, не говоря уже о запуске.

Большинство лицензий товарного типа часто называются EULA или лицензионными соглашениями конечного пользователя. Они обычно указывают, что вам разрешается делать с программным обеспечением (часто включая то, для чего вам разрешают его использовать). EULA чаще подчёркивают то, что запрещено, чем то, что позволяется. Одна из многих тем - перераспространение программного обеспечения. Большинство EULA явно отвергает перераспространение.

Linux (и свободное программное обеспечение в целом) отличается. Сопроводительная лицензия предоставляет право копировать программное обеспечение, получать исходный код, изменять  и перераспространять (с или без модификации) и даже продавать его. Поскольку существует множество вариантов, то возможно имеется и множество популярных лицензий.

\section{Какие существуют лицензии?}

Я перечислю несколько наиболее популярных лицензий, но уточню, что вокруг существует более 800 лицензий. Множество из тех лицензий довольно схожи (или точно такие же), и сообщество свободного программного обеспечения должно начать консолидировать все  эти лицензии в намного меньший набор. К сожалению, они этого пока так и не сделали. К счастью, здесь применяются правила 90-10: 90\% всего свободного программного обеспечения используют 10\% свободного программного обеспечения (или другой) лицензии. Другие лицензии используются лишь незначительно, иногда только для единственного приложения.

\subsection{Public Domain (Общественное достояние)}

Когда программное обеспечение размещено под общественным достоянием, Вы свободны  распоряжаться им вне зависимости от предпочтений: автор размахивает любым правом, которым он может для обеспечения полной свободы его программы.

\subsection{Лицензия MIT и некоторые подобные лицензии  BSD}

Лицензия MIT и некоторые подобные лицензии BSD почти похожи на Public Domain, однако просят сохранить уведомление об авторском праве без изменений. Это очень популярная лицензия, поскольку автор позволяет распоряжаться им вне зависимости от предпочтений, пока его имя сохраняется в примечаниях продукта об авторском праве.

\subsection{GPL}

Общественная Лицензия GNU - наиболее широко используемая лицензия свободного программного обеспечения, но для некоторых людей также самая строгая лицензия свободного программного обеспечения. GPL указывает, что можно делать с программным обеспечением, пока обеспечивается исходный код модификаций каждому, кому распространена изменённая версия и также пока эта модификация находится под GPL.

\textbf{Ядро Linux - лицензировано GPL}.

\section{Лицензии, утверждённые OSI}

Лицензия, утверждённая OSI - это лицензия, придерживающаяся определения Открытый исходный код, написанного по инициативе открытого исходного кода, следующие моменты которой являются свободной интерпретацией:

\begin{itemize}
	\item свободное перераспространение
	\item доступный исходный код
	\item модификации позволяется (включая перераспространение)
	\item никакой дискриминации (люди, сферы\ldots)
\end{itemize}

\section{Лицензии, утверждённые FSF}

Лицензия, утверждённая FSF придерживается определения Свободное программное обеспечение, написанного Фондом свободного программного обеспечения, следующие моменты которого являются ядром определения:

Вы должны быть свободны\ldots:

\begin{itemize}
	\item выполнять программу в любых целях
	\item изучать, как программа работает и адаптировать её по своим потребностям
	\item перераспространять копии
	\item улучшать программу и выпускать свои изменения в общественность
\end{itemize}

\section{Свободное программное обеспечение не является некоммерческим}

Свободное программное обеспечение часто воспринимается как проект человека, увлечённого своим хобби в чистом виде: оно не было бы коммерчески жизнеспособным, чтобы привнести свободное программное обеспечение в мир предприятий (enterprise). В конце концов, если оно находится в свободном доступе, какую прибыль компания могла бы получать. Ничто не могло быть таким далёким от истины\ldots

Это правда, что свободное программное обеспечение требует иного взгляда на программное обеспечение в коммерческой среде (включая компании). Компании, которые используют программное обеспечение, хотят быть уверенными в том, что у них имеется поддержка, когда что-либо идёт не так, как должно. Они часто закрывают (дорогостоящие) контракты на поддержку с компанией-разработчиком, где определены соглашения об уровне обслуживания (сокращённо SLA). На основе этих контрактов у компании имеется страхование, таким образом, если определённые услуги станут недоступными, поддерживающая компания предпримет всё что может, чтобы возвратить услугу, или в некоторых случаях, компенсировать финансовый ущерб, который нанесло банкротство.

Большую часть времени эти контракты на поддержку закрыты самой компанией-разработчиком, поскольку у неё имеется большая часть знаний касающихся программного обеспечения (поскольку, это вероятно, единственная компания с доступом к программному коду). К сожалению, по столь хорошей причине, как эта, компании не смотрят на свободное программное обеспечение, ''поскольку здесь нет никакой поддержки''. Что не является истиной; поддержка свободного программного обеспечения все ещё (коммерчески) доступна, но большую часть времени не от самих создателей. И несмотря на то, что это пугает компании, причина того, почему эта поддержка все ещё так же хороша как с массовым ПО, остаётся той же: поддерживающая компания имеет доступ к исходному коду инструмента и профессиональные знания об инструменте. У неё, вероятно, имеются разработчики в самом проекте программного обеспечения.

Компании, занимающиеся его продажей, конечно, часто против свободного программного обеспечения. Когда крупный доход этих компаний зависит от продаж их программного обеспечения, оно не было бы жизнеспособным, чтобы стать свободным. И если бы оно было, конкурирующие компании имели бы полный доступ к исходному коду и улучшали бы их собственный продукт с помощью него.

Тем не менее я не думаю, что это недостаток. Компании-разработчики программного обеспечения должны использовать свою основную силу: знания об инструменте. Как упомянуто прежде, другие компании часто хотят закрыть контракты на поддержку, чтобы гарантировать услугу, которую предоставляет программное обеспечение; если бы компания-разработчик создала свободное программное обеспечение, это не изменилось бы. Для многих компаний-разработчиков контракты на поддержку - основной источник дохода.

Все ещё возможно продать свободное программное обеспечение; некоторые новаторские компании оплачивают модификации, так как не имеют ресурсов, чтобы делать это самим. Эти компании могут сохранять модификации частными, если лицензия свободного программного обеспечения позволяет это), но могут также представить эти модификации общественности, внося их в сам проект программного обеспечения.

Главное доказательство этого - принятие свободного программного обеспечения основными игроками, такими как Sun Microsystems и IBM, появление новых игроков, которые создают свой бизнес на нем, такие как RedHat или MySQL (недавно приобретённый Sun Microsystems). Последняя компания использует подход двойного лицензирования: исходный код MySQL доступен в двух лицензиях, одно в качестве свободного программного обеспечения для общественности и другое более закрытое, для компаний, которым необходима поддержка со стороны самого MySQL. Использование подхода двойного лицензирования позволяет компании поддерживать фиксированное состояние их продукта при сохранении программного обеспечения в свободном виде. Поддержка фиксированного состояния продукта, конечно, намного проще, чем поддерживать программное обеспечение в целом.

Однако, не стоит думать, что каждый проект свободного программного обеспечения готов к предприятию, или, что Вы сможете найти (оплачиваемую) поддержку каждого такого проекта. Вы должны тщательно проверить каждый тайтл, если хотите использовать программное обеспечение, свободное или нет. Дистрибутивы для конечных пользователей помогают выбирать. Если дистрибутив упаковывает определённый тайтл, это означает, что он стабилен и хорошо поддерживается.

\section{Таким образом, Linux свободен?}

Да, Linux свободен. Он, конечно, свободен в смысле ''свободы слова'' и несмотря на то, что большинство тайтлов также бесплатно в смысле ''бесплатного пива'', Вы не должны удивляться, если увидите дистрибутивы, за которые можете или должны заплатить. В этом случае, вы можете платить за носитель программного обеспечения (записанный DVD), сопровождающую распечатанную документацию, 30-дневную установку и поддержку по  использованию или для ресурсов, которые дистрибутив должен получить сам (такие как инфраструктура).

Большинство дистрибутивов имеет возможность бесплатных загрузок с онлайн-документацией и замечательной общественной поддержкой (активные списки рассылки или интернет-форумы), что и является причиной такой популярности Linux: Вы можете загрузить, установить и использовать несколько дистрибутивов, чтобы решить, какой является лучшим. Вы можете попробовать программное обеспечение (без потери функциональности), и не обязаны даже платить за него, чтобы продолжать использовать (как это имеет место с условно-бесплатным программным обеспечением). Gentoo - один из таких проектов дистрибутива. Такие дистрибутивы получают свою финансовую поддержку (для инфраструктуры и организационных потребностей, включая юридическую поддержку и бюрократические документы) от пользовательских пожертвований или продаж отпечатанных DVD. Компании также склонны поддерживать дистрибутивы в финансовом отношении или аппаратными средствами/пропускной способностью.

Некоторые дистрибутивы становятся доступны, только когда вы платите за них. В этом случае, вы чаще всего платите за поддержку или дополнительное программное обеспечение в дистрибутиве, которое не находится в свободном доступе. Популярный дистрибутив - RedHat Enterprise Linux, дистрибутив Linux, в частности, предназначающийся для компаний, которые необходимо установить серверы Linux. Вы платите не только за поддержку, но также и за ресурсы, которые RedHat вложил в дистрибутив, чтобы сертифицировать его для другого программного обеспечения (такого как Oracle и SAP) таким образом, чтобы вы могли выполнять (с поддержкой со стороны компании-разработчика)  данное программное обеспечение на своих инсталляциях RHEL.

Однако, важно понять, что проекты дистрибутива  разрабатывают только очень небольшую часть программного обеспечения, которое вы устанавливаете в своей системе. Оное в большинстве прибывает из других проектов свободного программного обеспечения, и эти проекты часто не получают вознаграждений от проектов дистрибутива. Тем не менее они в действительности сталкиваются с теми же проблемами, как любой другой (большой) проект: бюрократические документы, юридическая поддержка, потребности инфраструктуры... Таким образом, нет ничего удивительного в том, что у этих проектов имеются те же потоки дохода, как и у проектов дистрибутива: пользовательские вознаграждения, коммерческое спонсорство и продажи программного обеспечения/поддержки.

\section{Модель разработки}

Ввиду особенности проектов свободного программного обеспечения, вы обнаружите, что у неё имеются некоторые различия с коммерческой рекламой закрытого исходного кода от готового программного обеспечения\ldots

\subsection{Многопроектная разработка}

Один дистрибутив обеспечивает агрегацию программного обеспечения. Каждый из этих тайтлов создан проектом программного обеспечения, который обычно отличается от проекта дистрибутива. Следовательно, когда вы устанавливаете дистрибутив на свою систему, он содержит программное обеспечение от сотен проектов во всем мире.

Таким образом, чтобы получить поддержку по дефекту, который нашли, или проблеме, с которой вы сталкиваетесь, первым местом для поиска поддержки стал бы дистрибутив, однако, вероятность состоит в том, что дистрибутив поместит вопрос о поддержке в апстрим. Это означает, что он передаст запрос в проект, разрабатывающий программное обеспечение, с которым у Вас имеется проблема.

\subsection{Прозрачная разработка}

Свободное программное обеспечение как правило разрабатывается прозрачно: если вы заинтересованы в разработке вашего любимого программного тайтла, то можете быстро выяснить, что она себя представляет и как принять в ней участие.

Традиционно, для хранения исходного кода программные проекты используют систему одновременных версий, такую как CVS или SVN. Подобные системы позволяют десяткам (или  даже тысячам) разработчиков одновременно работать над одним и тем же исходным кодом  и отслеживать все произведённые изменения (таким образом, они могут быть обращены). Это касается не только проектов свободного программного обеспечения — большинство из них использует такую систему. Тем не менее, проекты свободного программного обеспечения как правило позволяют людям, не являющимся разработчиками следить за прогрессом в разработке, предоставляя им доступ к системе в режиме «только для чтения». Это позволяет персонально отслеживать каждое изменение в программе. 

Большинство проектов программного обеспечения не могут использовать встречи в реальной жизни для обсуждения будущего программы или принятия решений по ее дизайну: их разработчики разбросаны по всему миру. Решением этой проблемы являются коммуникационные системы, такие как списки рассылок, IRC (чат) или форумы (Интернет или UseNet).  Большинство этих коммуникационных систем также открыты для обычных людей для участия в обсуждениях, что в свою очередь означает то, что конечные пользователи могут напрямую общаться с разработчиками. 

Последнее имеет важное преимущество: изменения, запрошенные пользователями, напрямую направляются разработчикам, что снижает частоту неверных толкований, позволяя проектам обновлять их программу наиболее аккуратно и чаще. 

\subsection{Циклы быстрых выпусков}

Большие проекты программного обеспечения имеют тысячи контрибьюторов и несколько десятков разработчиков. Разработчики значительно мотивируются энтузиазмом на разработку программы. В противном случае они не станут работать над программой по причине отсутствия прочих мотиваций (например  зарплаты, выдаваемой чеком), тем не менее, это должно быть сказано также про проекты (их не так уж и мало), которые платят разработчикам. В результате, программа быстро прогрессирует и обзаводится новыми возможностями (некоторые проекты даже имеют ежедневные выпуски, содержащие новые возможности). 

Чтобы удостовериться в корректности тестирования новых возможностей и исправлений, моментальные снимки разрабатываемых программ передаются сообществу тестеров, а стабильные выпуски выпускаются для общественности чаще в качестве нового выпуска программы. Различные виды выпусков обширно используются в среде проектов свободного программного обеспечения:

\begin{itemize}
	\item ночные  снимки являются выдержками из исходного кода за определённый период времени, которые собираются и выкладываются в онлайн для использования каждому желающему. Эти выпуски создаются автоматически и являются новейшими так как отображают состояние программного тайтла за определённый отрезок времени. Они очень экспериментальные и предназначены лишь для разработчиков или опытных контрибьюторов
	\item разрабатываемые выпуски  - это промежуточные выпуски, схожие с ночными, но в некоторой степени координируемые разработчиками. Как правило, они имеют журнал изменений, в котором приводятся изменения, внесённые с момента предыдущего выпуска. Такие выпуски предназначены для опытных тестеров, не возражающих против того, что время от времени программа может работать некорректно (иметь дефект).
	\item Бета-выпуски содержат предварительное состояние того, как должен выглядеть финальный выпуск. Он может быть неполностью стабильным или завершённым, однако отдельные лица, не участвующие в регулярных тестах могут опробовать и увидеть, как выглядел бы новый выпуск и что он содержит запрошенные ими исправления. Бета-выпуски также важны для дистрибутивов, так как они могут начать разработку пакетов для программы и подготовиться к финальному выпуску
	\item Релиз кандидаты — это кандидаты на финальный выпуск.  Содержат программу, которую разработчики желают выпустить. Они ожидают некоторый период времени, так как тестеры и широкая общественность могут провести свои тесты, чтобы убедиться в отсутствии в них ошибок и прочего. Теперь в программу добавляются только исправления, но не возможности. Если новых (или важных) ошибок  не обнаружено, кандидат на выпуск становится новым выпуском
	\item Стабильные выпуски — это финальные выпуски всего процесса разработки. Эти выпуски теперь используются пользователями и дистрибутивами, а весь процесс разработки начинается заново.
\end{itemize}

Стабильные выпуски также имеют обыкновение выпускаться в определённых градациях, что отражается их нумерацией версии. Популярной схемой нумерацией является \texttt{x.y.z}, где: 

\begin{itemize}
	\item \underline{\textbf{x}} --- основная версия; данная цифра версии обновляется в случае кардинальных изменений в программе.  Зачастую такие выпуски также требуют все зависимые от них пакеты для своего обновления, а также потому, что они могут использовать возможности или библиотеки, которые были изменены.
	\item \underline{\textbf{y}}  --- второстепенная версия;  данная цифра версии обновляется всякий раз, когда программа обновляется и получает множество новых особенностей
	\item \underline{\textbf{z}} — версия исправления;  данная цифра версии обновляется всякий раз, когда основные исправления добавляются в программу
\end{itemize}

В качестве примера, я перечислю дату выпусков KDE 4.9. Так как KDE является полноценной графической средой, её цикл выпусков «медленнее» других. Если вы сравните цикл выпусков, к примеру, с аналогичным циклом Microsoft Windows, то обнаружите, что он всё ещё молниеносно быстрый.  Конечно, это может быть похоже на сравнение яблок и апельсинов... но смотрите сами:
\begin{itemize}
	\item 2012-05-30: выпущен KDE 4.9 beta1
	\item 2012-06-14: выпущен KDE 4.9 beta2
	\item 2012-06-27: выпущен 1-й релиз кандидат KDE 4.9 
	\item 2012-07-11: выпущен 2-й релиз кандидат KDE 4.9
	\item 2012-08-01: выпущен KDE 4.9 
	\item 2012-10-02: выпущен KDE 4.9.2
	\item 2012-11-06: выпущен KDE 4.9.3
	\item 2012-12-04: выпущен KDE 4.9.4
	\item 2013-01-02: выпущен KDE 4.9.5
\end{itemize}

Просто, для справки, KDE 4.10 beta 1 выпущен 21 ноября 2012 года, всего через 6 месяцев после выпуска бета-версии KDE 4.9, а KDE 4.10 (теперь именуется «KDE Software Compilation — Набор приложений KDE», так как приложения для KDE самостоятельны и следуют своему циклу релизов) выпущен 6 февраля 2013 года, - через  месяцев после KDE 4.9.

\section{Большая документационная база}

Поскольку проект чаще всего не может предоставить людей и платную поддержку для ПО, его успех в большинстве своём основан на документации. Если сопроводительная документация содержит сведения о нём, опытные или независимые пользователи могут найти в ней ответы на все интересующие их вопросы.

Проекты свободного программного обеспечения обычно имеют документацию высокого профиля, которая чаще всего оказывается лучше чем документация закрытого ПО, доступная в сети. Множество больших проектов имеют в наличии всю документацию, доступную  даже на нескольких языках. И если вы не нашли ответа в документации проекта, есть шансы, что где-либо один или несколько пользователей написали независимое руководство по этому ПО.

В Интернете существует множество сайтов, ссылающихся на различные ресурсы с документацией и возникает такая же проблема как и с самим СПО: зачастую у вас имеется слишком большое количество ресурсов, что затрудняет в поиске правильного документа, который мог бы проинструктировать вас по программе через получение пользовательского опыта. Однако, в отличие от изобилия тайтлов ПО (усложняющего поиск правильного ПО, соответствующего задаче) пользователю легче узнать хороша документация или нет, таким образом нет необходимости в 'её распространении'.

\section{Жизненный цикл ПО}

Если вы приобрели ПО у маленькой, неизвестной компании, может получиться так, что через много лет этой компании больше не станет, как и поддержки самого продукта с этого момента. Что-то похожее является истиной и в отношении   свободного ПО: если проект решает, что ему не хватает ресурсов для продолжения разработки (обычно по причине нехватки разработчиков), он прекращает разработку, что обычно отражается и на прекращении поддержки от пользователей.

Тем не менее, в отличие от случая с продуктом компании, исходный код свободного ПО остаётся доступным для общественности. Если вам крайне необходимо заставить ПО работать на вас, вы можете получить исходный код и продолжить его разработку самостоятельно (или заплатить другим, чтобы они выполнили это для вас). Вы также можете быть уверены, что ПО останется свободным. 

Если владельцы авторского права вдруг решат, что ПО находится под другой лицензией, с которой вы в дальнейшем не согласны, вы можете получить исходный код прежде чем владельцы авторского права решат сменить лицензию и продолжить разработку в рамках этой лицензии (когда ПО все еще находится под оригинальной лицензией, но не новой) Этот процесс (когда группа разработчиков несогласна с планом разработки ПО и начинает новый проект, основанный на том же исходном коде) называется форк проекта.

Наиболее известный пример форка — создание проекта X.org, форка другого проекта под названием XFree86, который в определённый момент решил сменить лицензию. Не только это было основанием для форка: некоторые разработчики были неудовлетворены политикой разработки касательно новых возможностей ПО и её темпом. Оба проекта всё ещё существуют, однако X.org отныне наиболее популярный из них.

\section{Открытые стандарты}

Поскольку в разработку вовлечено множество проектов, важно, чтобы они как можно больше следовали стандартам. Только соответствие открытым стандартам позволяет проектам легко и эффективно работать вместе. Далее приводятся несколько важных стандартов или наиболее заметных спецификаций в мире свободного ПО.

\section{Стандарт иерархии файловой системы (FHS)}

Первым стандартом с которого я начну обсуждение является Стандарт иерархии файловой системы, сокращённо FHS. Этот стандарт используется почти всеми дистрибутивами и описывает расположение файлов в файловой системе Linux. Об этом можно прочитать по адресу \href{http://www.pathname.com/fhs/}{http://www.pathname.com/fhs}, однако многие другие ресурсы также описывают структуру FHS.

Структура файловой системы для систем Unix/Linux достаточно отличается от структуры файловой системы, замеченной в Microsoft Windows. Вместо маркировки разделов по букве, системы  Unix/Linux отображают файловую системы в виде древовидной структуры, начиная от корневого каталога и  дадее построением каталогов и файлов. Вы можете назвать ветки в этой структуре каталогами, которые оканчиваются файлами. Если вы думаете, что ранее не встречались с файловой системой Unix/Linux, то подумайте ещё раз: URL-адреса, которые вы используете в Интернете основаны на её структуре. К примеру: URL-адрес \texttt{http://www.gentoo.org/doc/en/faq.xml} указывает на файл faq.xml, который расположен на сервере \href{www.gentoo.org}{http://www.gentoo.org}, в каталоге /doc/en. Таким образом, / - это корневой каталог, «doc» - ветка этого каталога, а «en» - ветка, принадлежащая ветке «doc».

Дистрибутивы, соответствующие FHS, позволяют пользователям легко переключаться между ними: структура файловой системы остается прежней, таким образом навигация между папками, файлами устройств …. не меняется. Это дает возможность упаковщикам создавать пакеты для нескольких дистрибутивов (до тех пор пока они используют тот же формат пакета). Но прежде всего, это позволяет пользователям одного дистрибутива Linux помочь пользователям других дистрибутивов, так как между структурой их файловых систем нет различий.

Текущая версия стандарта — 2.3, выпущена 29 января 2004 года.

\section{Linux Standard Base}

Linux Standard Base или LSB состоит из структуры, двоичной совместимости, необходимых библиотек и команд и др. для операционной системы Linux. Если дистрибутив соответствует стандарту LSB, он способен устанавливать, запускать и сопровождать LSB-совместимые (программные) пакеты.

Дистрибутивы должны соответствовать LSB если для них необходимо удостовериться в том, что они не отклоняются от этого стандарта. В последующем, LSB является попыткой удостовериться, что дистрибутивы остаются схожими с учётом библиотек, команд, или пользовательского опыта в общем. Это хорошее стремление с целью обеспечить отсутствие фрагментации в мире Linux.

Поскольку LSB является мостом, объединяющим другие стандарты, включая вышеупомянутый FHS, а также Единую спецификацию Unix (SUS), которая определяет, какой должна быть система Unix. Однако, нельзя сказать, что ОС Linux является Unix, так как нам необходимо сертифицировать её (что требует серьёзную финансовую поддержку) и эта сертификация просуществует недолго вследствие частых изменений в системе Linux.

\section{Спецификации Free Desktop}

По адресу  \href{http://www.freedesktop.org}{http://www.freedesktop.org} вы найдёте набор спецификаций для рабочего стола, которые также широко известны в сообществе свободного программного обеспечения. Несмотря на то, что они не являются стандартами (так как freedesktop не является их представителем и эти спецификации не конвертированы стандарты OASIS или ISO) многие дистрибутивы соблюдают их.

Эти спецификации определяют, как создаются и сопровождаются записи в меню, где должны размещаться значки, как должно происходить перемещение между различными библиотеками (больше относится к Qt и GTK+, графическим библиотекам KDE и GNOME).

\newpage
{\color{white}\section{Упражнения}}
\begin{tcolorbox}[title=\textbf{Упражнения}, colback=yellow!14!white, colframe=red!75!white]
\begin{enumerate}
	\item В чём отличие между GPLv2 и GPLv3?
	\item Часть стандарта LSB  - ELF или формат исполняемых и компонуемых файлов, скомпилированный код которого используется различными дистрибутивами Linux/Unix. Вы можете назвать операционную систему, поддерживающую формат ELF кроме Linux/Unix?
	\item Некоторые пользователи считают быстрые выпуски слабостью сообщества свободного программного обеспечения: они «вынуждены» обновлять своё ПО очень часто, даже если оно свободно это всё равно занимает некоторое время (а иногда является головной болью). Некоторые дистрибутивы пытаются помочь пользователям, предлагая только стабильное ПО (стабильное как само по себе так и в выпусках версий). Как это возможно?
	\item Как стало возможным то, что многие дистрибутивы позволяют обновлять ПО до последних версий без необходимости установочного CD или переустановки с нуля?
\end{enumerate}
\end{tcolorbox}

\phantom{}
\begin{tcolorbox}[title=\textbf{Дальнейшие ресурсы}, colback=yellow!14!white, colframe=red!75!blue]
\begin{itemize}
	\item[+] Собор и Базар  [\href{http://www.catb.org/~esr/writings/cathedral-bazaar/cathedral-bazaar/}{http://www.catb.org/~esr/writings/cathedral-bazaar/cathedral-bazaar/}], автор Эрик Стивен Рэймонд — эссе на тему о двух различных моделях разработки, используемых в сообществе Свободного Программного обеспечения.
	\item[+] Фонд Свободной  информационной инфраструктуры  [\href{http://www.ffii.org}{http://www.ffii.org}] — некоммерческая организация, нацеленная на установление свободного рынка в информационных технологиях.
	\item Битва за программные патенты [\href{http://www.gnu.org/philosophy/fighting-software-patents.html}{http://www.gnu.org/philosophy/fighting-software-patents.html}], автор Ричард Столлман — взгляд фонда GNU на программные патенты.
\end{itemize}
\end{tcolorbox}

\newpage

%%% Глава 3: Роль сообщества %%%

\chapter{Роль сообщества}

\section*{Сообщества}

Самое важное достоинство СПО заключается в его сообществе. Как и любая технология или концепт, СПО имеет своих приверженцев, защищающих и возвеличивающих его. Сообщество Свободного Программного Обеспечения (далее «сообщество СПО», прим. переводчика) само по себе очень яркое, живое и жаждет помочь другим открыть удивительный мир СПО\ldots

\section{Сообщества}

Сообщества СПО схожи с реальными сообществами, но используют Интернет как главный коммуникационный узел. Следовательно, они не группированы в сети подобно реальным, однако разбросаны по миру. Несмотря на это, Интернет гарантирует, что участники сообщества общаются с другими также как и с соседями, даже если находятся на расстоянии световых лет (фигура речи).

Интернет представляет значительную ценность для этих сообществ: вас оценивают не по цвету кожи, возраста или по тому как вы выглядите. Важно то, как вы взаимодействуете с другими, как вы представляете себя и откликаетесь в обсуждениях. Споры в сообществе часто могут быть довольно жаркими, особенно когда в факты в теме беседы недостаточны для получения хорошего ответа. И когда споры сменяются почти обменом оскорблений, рождается флейм.

В этих случаях факты и доводы обычно далеки. Вам однозначно стоит избегать флейма в беседах, где могут приниматься решения, но его действительно невозможно предотвратить, так как это результат деятельности людей, наиболее заинтересованных в теме, которую они защищают, особенно когда нет точного ответа на флеймообразующий вопрос. 

Примерами таких флеймов являются вопросы «Какой самый лучший дистрибутив Linux» или «Какой текстовый редактор мне стоит выбрать», поскольку на эти вопросы нет точного ответа: дистрибутив, наиболее хорошо подходящий одному, может быть худшим для другого, то же самое можно сказать и о текстовых редакторах. На латиноамериканском языке можно сказать ''de gustibus et coloribus non est disputandum'' (что на русский язык можно перевести как «на вкус и цвет мастера нет» прим. переводчика) и это очень верно описывает характер этих вопросов. 

Когда у вас нет выбора, нет и флейма: нельзя сравнить один продукт сам с собой. Однако в мире СПО, выбор — важное понятие. У вас есть выбор среди множества свободных операционных систем (кроме Linux есть множество разновидностей BSD, Sun Solaris 10 и даже менее популярных, но обещающих систем, таких как GNU Hurd), дистрибутивов (доступно свыше ста дистрибутивов), графических сред (ни дня не проходит без священных войн  а-ля GNOME против KDE), офисных пакетов и т.д.

«Лучший дистрибутив» - это неоднократно оспариваемая тема и тем не менее эта книга оказывает влияние на нее. Поэтому лучшим моим ответом для вас будет то, что лучшего дистрибутива не существует, по крайней мере не в общих чертах. Понятие «лучший» рассматривается теми людьми, у которых есть свои предпочтения на счёт операционной системы. И множество этих людей очень яростно защищает свой дистрибутив. 

Сообщества дистрибутивов очень активны, так как они достаточно большие. Сообщество Gentoo известно своей отзывчивостью: канал чата Gentoo не умолкает ни минуты (более 800 участников в любое время) как и форум (свыше тысячи комментариев в день) и списки рассылок. Конечно, общий флейм обычно происходит на нейтральной стороне, но ежедневны и горячие споры на другие темы. 

По этой причине большинство сообществ имеют людей, ответственных за здоровую атмосферу в беседах, и предотвращающих рост флейма. Те, кто провоцируют флейм на канале связи (называются троллями) отслеживаются операторами: они могут выгнать или даже «забанить» нарушителя правил с канала чата, операторы списка рассылок удаляют их из списка, а модераторы форумов удаляют их профили. Можно сказать, что эти люди играют роль полицейских в сообществе. 

\section{Местные сообщества}

Особый вид сообщества в каком-либо регионе. Такие сообщества нередко организовывают встречи (конференции, беседы, барбекю и т.п.) и предлагают помощь населению по месту своего нахождения. 

ГПЛ (Группы пользователей Linux) являются успешными примерами таких сообществ: эти группы объединяются вместе, обсуждают развитие в мире Linux и помогают другим участникам в установке Linux-систем (Инсталфесты Linux — это местные встречи, участники которых оказывают помощь в установке вашего любимого дистрибутива на компьютер). Вы можете обнаружить ГПЛ в своей местности. 

Множество ГПЛ предлагает различные услуги для своих пользователей, которых обычно не встречается в сообществах коммерческого ПО. Более того, они предоставляют эти услуги безвозмездно.

\begin{itemize}
	\item Индивидуальная, локальная помощь в установке, настройке и обслуживании дистрибутива Linux или другого СПО
	\item Курсы, беседы и презентации, предлагающие более глубокое погружение в мир доступного СПО
	\item Специфичная документация, предназначенная для нужд её пользователей
\end{itemize}

Если у вас есть время, которое вы можете уделить этому, я настоятельно советую присоединиться к местной ГПЛ — даже если вы не ищете помощи, вы все ещё можете предложить свои опыт и знания другим  обрести связи (да, социальные сети важны).

\section{Сообщества в сети}

Когда пользователи хотят обсудить на тему конкретного ПО или дистрибутива, то это нередко приводит к формированию онлайн-сообществ. Эти сообщества не организовывают (или лишь немного расширяют) встречи в определённом регионе (часто «в реальной жизни») но по большей части используют Интернет в качестве площадки для встреч («онлайн»-встречи).

Онлайн-сообщества располагают преимуществом, заключающимся в том, что их члены могут быть в любой точке мира, и они, также как и ГПЛ, всё ещё предоставляют помощь своим пользователям, в большинстве случаев на безвозмездной основе:

\begin{itemize}
	\item онлайн-помощь с установкой, настройкой и обслуживанием СПО
	\item В конкретных случаях, сообщества даже могут предоставить интерактивную помощь с использованием технологий, таких как SSH (Secure SHell - безопасный командный процессор, позволяющий пользователям получать доступ и работать в другой системе) и VNC (Virtual Network Computing — система удалённого администрирования,  позволяющая пользователям получать доступ и работать в другой системе, используя графику, или просматривать сеансы в режиме «только для чтения»).
	\item курсы и онлайн-презентации
	\item документация, более специализированная для программных тайтлов, однако также нередко и локализованная (переведённая)
\end{itemize}

Это стало возможным благодаря различным технологиям, доступным в Интернете, включая:

\begin{itemize}
	\item Вики (программное обеспечение с возможностью совместной разработки документации в сети) - это ПО стало достаточно популярным для разработки и выпуска документации. Его использование позволяет пользователям редактировать существующую документацию или стать авторами новой (с простым обзором), а результаты их деятельности немедленно станут доступны для других.
	\item (Веб)форумы, где люди могут принять участие в обсуждениях, размещая свои сообщения и отвечая на другие. Преимущество таких форумов заключается в том, что они доступны через ваш веб-браузер (которые все еще допускаются большинством фаерволов), информацию можно получить спустя продолжительное время после закрытия обсуждения, а сообщения могут быть дополнены изображениями, вложениями и отформатированным текстом.
	\item Списки рассылок, схожие (по функциональности) с веб-форумами, но организованные через электронную почту. Люди подписываются на список рассылки,  принимают все письма, отправленные в список рассылки, на свою личный почтовый ящик. Ответы на эти письма отсылаются обратно в списки рассылок, где они снова распространяются на всех участников списков рассылок. Списки рассылок достаточно популярны в сообществе СПО, так как легко модерируются и могут быть отфильтрованы. Также, такие письма достигают адресатов быстрее, чем сообщения на веб-форумах, таким образом, вы можете рассматривать список рассылки как платформу мгновенных обсуждений.
	\item IRC (Ретранслируемый интернет-чат) интерактивный способ связи с множеством людей. ПО для мгновенных сообщений многим известно как MSN или Google Talk. IRC — это нечто старое, но всё ещё активно используемое, поскольку оно поддерживает чат-комнаты, где могут участвовать несколько сотен человек. IRC — это наиболее быстрая платформа для участия в беседах и может быть рассмотрена в качестве способа создания «онлайн» встреч.
	\item Социальные платформы, ориентирующиеся на людей, такие как Google Plus или Facebook, люди, имеющие общие интересы сотрудничают и ведут беседу на любимые темы. Их преимущество в том, что они лучше интегрированы с последними технологическими достижениями; люди с современными мобильными телефонами (смартфонами) доступны в сети продолжительное время и быстро взаимодействуют со всем что случается в сообществах.
\end{itemize}

\section{Поддержка}

Сообщества обычно выполняют роль поддержки пользователей: если у вас есть вопросы об их проекте ПО, они захотят ответить и оказать помощь. Если вы считаете, что ПО имеет недостаточную функциональность, они помогут вам дополнить её или заставить работать вместе с другими инструментами (или даже направить вас на другие проекты ПО, если они чувствуют, что вам нужно что-то, для чего их любимый инструмент не был создан).

Поддержка может быть оказана на многих этапах\ldots

\section{Справочники по документации}

Справочник по документации обычно создаётся с одной целью: описать как сделать что-то с инструментом. Такие справочники тем не менее чаще называются HOWTO (пошаговые инструкции). В них вкладывается много усилий, так как они должны быть корректными, хорошо оформленными и завершёнными. Хорошее HOWTO — это то, после прочтения которого у вас останется меньше вопросов.  Если вы спросите сообщество как сделать что-либо нужное вам и это описано в HOWTO, то вас направят к нему (иногда с более грубой отсылкой на термин RTFM или «Read The Fucking Manual» (что означает «Читайте чёртову инструкцию» - однако третье слово после артикля может читаться как «Fine» - «Хороший»).

Прочие виды документации — ЧАВО (Часто Задаваемые Вопросы), которые в общем виде являются весьма небольшими HOWTO или ответами скорее на концептуальные вопросы, чем технические. Когда вы впервые столкнулись с определенным инструментом, вам стоит прочитать ЧАВО прежде чем задать вопрос. Высокая вероятность заключаются не только в том, что вы найдёте ответ на свой вопрос, вы можете узнать больше об интересующем вас инструменте. 

Некоторые сообщества предлагают базу знаний. Такие системы могут рассматриваться как сборник вопросов и ответов, но в отличие от ЧАВО они не дают ответы на частые вопросы. Базы знаний чаще всего предлагают решения для специфичных настроек. 

\section{Интернет и форумы Usenet}

Интернет-форумы (веб) или форумы Usenet (пользовательская сеть) являются наиболее интерактивным подходом к получению поддержи. Интернет-форумы имеют дополнительное преимущество в том, что вы можете придать тексту своих вопросы определённое форматирование: вы можете показать код команды, исключения или ошибки лучше, чем простым тексом. Вы даже можете добавить скриншоты. Форумы позволяют любому пользователю достаточно быстро получить помощь: форумы читаются многими людьми и их внешний вид достаточно прост для просмотра новых тем. 

Ещё одно дополнительное преимущество интернет-форумов заключается в том, что как только ответ на вопрос получен, он сохранится в базе данных форума.

Следовательно, весь форум может рассматриваться как база знаний с множеством ответов. Наиболее популярные темы обычно прикрепляются, что означает, что тема останется в первом ряду даже без возможности обсуждения, увеличивая вероятность её прочтения новыми пользователями.

Форумы Usenet (или новостных групп) ещё один способ поддержки, однако, нужно отметить, что новостные группы редко используются для средств СПО. Обычно вы найдёте новостную группу в случае если проект не предоставляет свой форум. (любой может запустить новую новостную группу) однако это приводит к тому, что интернет-форумы и форумы Usenet оказываются связаны: комментарии с одного форума сливаются с другим.

\section{Списки рассылок}

Наиболее прямой способ — это списки рассылок, где индивидуально просматриваются несколько десятков (или даже сотни) e-mail адресов. Список рассылки нередко оказывается быстрее форумов, поскольку множество разработчиков их завсегдатаи из-за их простоты: списки рассылок представляют собой обычный текст, который можно легко фильтровать.

Большинство списков рассылок также архивируется, позволяя вам бегло пробежать по старым темам в списке. Они используются как первичный канал связи для нужд разработки, тогда как форумы подходят для пользователей и их опыта. Некоторые проекты имеют внутренние списки рассылок для разработчиков и они недоступны для общественности. Не потому, что они хотят скрыть от пользователей результаты разработки: такие списки рассылок используются для связи по проблемам безопасности, персональной информации и общения на темы, которые являются юридически сложными для защиты, в случае если их обнародуют.

\section{Чат}

Чат-беседа — в большинстве случаев почти прямая форма связи с другими участниками. Множеств проектов СПО использует IRC (ретранслируемый интернет-чат) в качестве центрального канала связи. Пользователи могут легко получать помощь через IRC, где разработчики общаются и обсуждают изменения.

Чат-каналы могут быть очень популярны. Главный чат-канал Gentoo (\#gentoo в сети freenode) может содержать от 800 до 1000 участников в любое время.

\section{Встречи в реальной жизни}

Периодически группы разработчиков собираются вместе для поддержки «в реальной жизни» или обсуждения развития их ПО. Во многих случаях, такие встречи предоставляют людям возможность получить интерактивную помощь. Мы обсуждали встречи ГПЛ (где встречи в реальной жизни — частое явление), однако сообщества ПО также организовывают встречи в реальной жизни. Большинство из подобных встреч предлагает способ при котором разработчики встречаются с друг-другом (впервые), обсуждают темы и учатся друг у друга.

В некоторых случая организуются хакфесты. Во время этих встреч, разработчики объединяются вместе с одной целью: разработать новые возможности или устранить ошибки в их ПО. Несмотря на это, это возможно и вне сети, хакфесты позволяют разработчикам свободно связываться с друг-другом и помочь другим разработчикам в их проблемах. Встречи в реальной жизни позволяют разработчикам легко обнаружить имеющуюся проблему (некоторые проблемы могут быть слишком сложными или отнимать много времени для их документирования).

\section{Конференции}

В мире СПО конференции организуются часто. Во время этих конференций

\begin{itemize}
	\item ведутся беседы об определённых тайтлах ПО ( на тему дизайна, возможностей, развития и т.д.) или о проекте (инфраструктура, предоставляемые услуги, используемые технологии и т.д.)
	\item организуются выставочные стенды, где возможности проектов будут представлены широкой публике. Для дистрибутивов часто используются такие стенды для демонстрации процесса установки с CD/DVD носителей и показать системы с запущенными дистрибутивами.
	\item Компании предоставляют сведения о том, как они используют (или разрабатывают) СПО (и иногда привлекают разработчиков)
\end{itemize}

\subsection{FOSDEM}

FOSDEM или Европейская конференция разработчиков свободного и открытого ПО проходит в Брюсселе, Бельгия, в начале каждого года (примерно в середине февраля). Во время этой конференции обсуждается написание кода и разработка ПО, однако вы найдёте выставки, посвящённые различным проектам ПО/дистрибутивов или комнаты разработчиков (где каждый отдельный проект может предложить беседу на темы, специфичные для проекта).

FOSDEM проходит всего два дня и стал главной конференцией в сообществе Свободного Программного Обеспечения особенно в Европе, так как множество других конференций располагаются в США.

\subsection{FOSS.IN}

FOSS.IN или Конференция, посвящённая свободному и открытому ПО в Индии  одна самых больших конференций FOSS Азии. Она проводится в конце каждого года в Бангалоре, Индия, в рамках которой ведутся беседы, обсуждения, симпозиумы, встречи и т.д. от международных докладчиков и разработчиков.

\subsection{LinuxTag}

LinuxTag — это выставка свободного ПО, главном образом фокусирующаяся на операционных системах  и решениях, основанных на Linux. В отличие от FOSDEM, LinuxTag фокусируется больше на интеграции Linux (и свободного ПО) в больших средах, предлагая выставки для коммерческих компаний и некоммерческих организаций.

Её слоган «Where ``.COM'' meets ``.ORG''» («Где компании встречаются с организациями»). Вы можете посещать LinuxTag ежегодно весной.

\newpage
{\color{white}\section{Упражнения}}
\begin{tcolorbox}[title=\textbf{Упражнения}, colback=yellow!14!white, colframe=red!75!white]
\begin{enumerate}
	\item Попробуйте найти способы обсуждений в сети (веб-форумы, списки рассылок, IRC)
	\end{enumerate}
\end{tcolorbox}

\phantom{}
\begin{tcolorbox}[title=\textbf{Дальнейшие ресурсы}, colback=yellow!14!white, colframe=red!75!blue]
\begin{itemize}
	\item[+] The Ottawa Linux Symposium [\href{http://www.linuxsymposium.org}{http://www.linuxsymposium.org}] проводится ежегодно в Оттаве, Канада, во время летних каникул.
	\item[+] Linux Kongress [\href{http://www.linux-kongress.org}{http://www.linux-kongress.org}] почти всегда проводится в Германии, но несмотря на это, один раз мероприятие проводилось в Англии.
	\item[+] Linux.conf.au [\href{http://linux.conf.au/}{http://linux.conf.au/}] проходит в Австралии в начале каждого года.
	\item[+] Ohio Linux Fest [\href{http://www.ohiolinux.org/}{http://www.ohiolinux.org/}] проводится в Огайо каждой осенью.
	\item[+] Linux Fest Northwest [\href{http://www.linuxfestnorthwest.org/}{http://www.linuxfestnorthwest.org/}] проводится в Вашингтоне каждой весной.
	\item[+] SCaLE (Southern California Linux Expo) [\href{http://scale7x.socallinuxexpo.org/}{http://scale7x.socallinuxexpo.org/}] выставка проводится поздней зимой в Южной Калифорнии.
	\item[+] Ontario Linux Fest [\href{http://onlinux.ca/}{http://onlinux.ca/}] ЛинуксФест в Онтарио
	\item[+] Конференция LinuxWorld и Expo [\href{http://www.linuxworldexpo.com/}{http://www.linuxworldexpo.com/}]
	\item[+] Freed.IN [\href{http://freed.in/}{http://freed.in/}]
\end{itemize}
\end{tcolorbox}
	
\end{document}