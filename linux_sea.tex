\documentclass[12pt]{book}
\pagestyle{headings} % для более удобной навигации
\usepackage[utf8]{inputenc} % чтобы не глючило. Собирается через pdfTeX!!!
\usepackage[english,russian]{babel} % 2 языка
\usepackage[margin=2cm]{geometry} % стандартная ширина поля в книгах
\usepackage{tcolorbox} % будет полезно
\usepackage{color} % будет полезно
\usepackage{droid} % вполне неплохо смотрится
\usepackage{graphicx}
\usepackage{pdfpages} % для вставки pdf при необходимости
\usepackage{hyperref}
\usepackage{indentfirst} % для отступов в основном тексте

%%% Поле для настройки отдельных %%%
%%  Элементов	ДО текста          %%
%\renewcommand{\abstractname}{О Книге}
%%%%%%%%%%%%%%%%%%%%%%%%%%%%%%%%%%%%
\begin{document}
%%% Поле для настройки отдельных %%%
%%  Элементов	по ходу             %%

%%% Оформление титульного листа %%%%
%   Титульный лист будет присое-  %%
%   диняться к файлу через pdf-   %%
%   shuffler, чтобы сработали     %%
%   все необходимые ссылки			  %%
%%%%%%%%%%%%%%%%%%%%%%%%%%%%%%%%%%%%

%%%   Страница с информацией     %%%
\pagestyle{empty}
\textbf{Мир Linux}

\phantom{}
\textbf{Свен Вермален}

\phantom{}
Copyright © 2009-2013 Sven Vermeulen

\phantom{}
Перевод с английского: Галым Керимбеков\footnotetext{\href{kerimbekov.galym@yandex.ru}{$^{*}$kerimbekov.galym@yandex.ru}}$^{*}$

\phantom{}
Пост-верстка: Stas Bookman\footnotetext{\href{nodor4mint@yandex.ru}{$^{**}$nodor4mint@yandex.ru}}$^{**}$
%%% Аннотация %%%
\begin{tcolorbox}[colback=cyan!22!white]
\noindent Книга "Мир Linux" предлагает мягкое, но не лишенное технических аспектов (с точки зрения конечного пользователя) знакомство с операционной системой на примере дистрибутива Gentoo Linux. Книга не станет описывать историю ядра Linux или дистрибутивов, либо погружать в детали, менее интересные для пользователей.

\phantom{}
\noindent Онлайн-руководство Gentoo предлагает весьма подробный подход по ряду разделов и поэтому обязательно к чтению любым пользователем, желающим знать обо всех возможностях этой операционной системы. Несмотря на то, что издание "Мир Linux"{} и онлайн-руководство Gentoo определённо пересекаются между собой, книга ни в коем случае не призвана заменить последнее.

\phantom{}
\noindent "Мир Linux"{} попытается сосредоточиться на темах, о которых повседневные пользователи, вероятно, должны знать, чтобы продолжать работать с Gentoo Linux.

\phantom{}
\noindent Версия, которую вы читаете в настоящее время, является v1.17 и была сгенерирована 18.02.2016 г. Доступны также версии ODT и ePUB.
\end{tcolorbox}

%%%  Оглавление  %%%
\tableofcontents
%%%%%%%%%%%%%%%%%%%%%%%%%%%%%%%%%%%%%
\end{document}